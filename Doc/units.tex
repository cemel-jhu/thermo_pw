%!
%! Copyright (C) 2019-2021 Andrea Dal Corso 
%! This file is distributed under the terms of the
%! GNU General Public License. See the file `License'
%! in the root directory of the present distribution,
%! or http://www.gnu.org/copyleft/gpl.txt .
%!
\documentclass[12pt,a4paper,twoside]{report}
\def\version{1.6.1}

\usepackage[T1]{fontenc}
\usepackage{tcolorbox}
\usepackage{bookman}
\usepackage{html}
\usepackage{graphicx}
\usepackage{fancyhdr}
\usepackage[Lenny]{fncychap}
\usepackage{color}
\usepackage{geometry}
\usepackage{amsmath}
\usepackage{mathtools}

\tcbuselibrary{breakable}
\pagestyle{fancy}
\lhead{Units guide}
\rhead{}
\cfoot{\thepage}

\newgeometry{
      top=3cm,
      bottom=3cm,
      outer=2.25cm,
      inner=2.75cm,
}

\definecolor{web-blue}{rgb}{0,0.5,1.0}
\definecolor{steelblue}{rgb}{0.27,0.5,0.7}
\definecolor{coral}{rgb}{1.0,0.5,0.3}
\definecolor{red}{rgb}{1.0,0,0.0}
\definecolor{green}{rgb}{0.,0.5,0.0}
\definecolor{dark-blue}{rgb}{0.,0.0,0.6}
\definecolor{limegreen}{rgb}{0.19,0.8,0.19}
\definecolor{orange}{rgb}{1.0,0.44,0.0}
\definecolor{violet}{rgb}{0.50,0.33,0.9}
\definecolor{light-yellow}{rgb}{0.94,0.85,0.62}

\tcbset{colback=light-yellow,colframe=dark-blue,breakable}

\def\qe{{\sc Quantum ESPRESSO}}
\def\pwx{\texttt{pw.x}}
\def\phx{\texttt{ph.x}}
\def\configure{\texttt{configure}}
\def\PWscf{\texttt{PWscf}}
\def\PHonon{\texttt{PHonon}}
\def\tpw{{\sc Thermo\_pw}}
\def\make{\texttt{make}}


\def\hplanck{6.62607015\times 10^{-34}}
\def\hbarf{1.0545718176462\times 10^{-34}}
\def\cspeed{2.99792458\times 10^{8}}
\def\e{1.602176634\times 10^{-19}}
\def\rydberg{1.0973731568160\times 10^{7}}
\def\alphaf{7.2973525693\times 10^{-3}}
\def\amu{1.66053906660\times 10^{-27}}
\def\avonum{6.02214076\times 10^{23}}
\def\kb{1.380649\times 10^{-23}}
\def\rgas{8.31446261815324}

\def\me{9.1093837015\times 10^{-31}}
\def\abohr{5.29177210903\times 10^{-11}}
\def\muzero{1.25663706212\times 10^{-6}}
\def\epsilonzero{8.8541878128\times 10^{-12}}
\def\ehartree{4.3597447222072\times 10^{-18}}
\def\bohrmag{9.2740100783\times 10^{-24}}

\def\barl{5.29177210903\times 10^{-11}}
\def\barm{9.1093837015\times 10^{-31}}
\def\barrhom{6.1473168257}
\def\bart{2.4188843265857\times 10^{-17}}
\def\barnu{4.1341373335182\times 10^{16}}
\def\barv{2.18769126364\times 10^{6}}
\def\bara{9.0442161272\times 10^{22}}
\def\barp{1.99285191410\times 10^{-24}}
\def\baram{1.0545718176462\times 10^{-34}}
\def\barf{8.2387234982\times 10^{-8}}
\def\baru{4.3597447222072\times 10^{-18}}
\def\barw{1.8023783420686\times 10^{-1}}
\def\barpr{2.9421015696\times 10^{13}}
\def\bari{6.623618237510\times 10^{-3}}
\def\barc{1.602176634\times 10^{-19}}
\def\barrho{1.08120238456\times 10^{12}}
\def\barcur{2.36533701094\times 10^{18}}
\def\bare{5.14220674763\times 10^{11}}
\def\barphi{2.7211386245988\times 10^{1}}
\def\barcap{5.887890530517\times 10^{-21}}
\def\bardip{8.4783536255\times 10^{-30}}
\def\barpolar{5.7214766229\times 10^{1}}
\def\bard{4.5530064316}
\def\barohm{4.1082359022277\times 10^{3}}
\def\barb{2.35051756758\times 10^{5}}
\def\barav{1.24384033059\times 10^{-5}}
\def\barwb{6.5821195695091\times 10^{-16}}
\def\bary{9.937347433815\times 10^{-14}}
\def\barmu{1.85480201567\times 10^{-23}}
\def\barmag{1.25168244230\times 10^{8}}
\def\barh{9.9605723936\times 10^{6}}

\def\cmtom{1.0\times 10^{-2}}
\def\gtokg{1.0\times 10^{-3}}
\def\ptop{1.0\times 10^{-5}}
\def\ltol{1.0\times 10^{-7}}
\def\ftof{1.0\times 10^{-5}}
\def\utou{1.0\times 10^{-7}}
\def\prtopr{1.0\times 10^{-1}}
\def\chtoch{3.33564095107\times 10^{-10}}
\def\kappaa{1.000000000274\times 10^{-1}}
\def\kappa{2.99792458082\times 10^{9}}
\def\kappadiecic{1.00000000027}
\def\itoi{3.33564095107\times 10^{-10}}
\def\rhotorho{3.33564095107\times 10^{-4}}
\def\curtocur{3.33564095107\times 10^{-6}}
\def\etoe{2.99792458082\times 10^{4}}
\def\phitophi{2.99792458082\times 10^{2}}
\def\captocap{1.11265005544\times 10^{-12}}
\def\diptodip{3.33564095107\times 10^{-12}}
\def\polartopolar{3.33564095107\times 10^{-6}}
\def\dtod{2.65441872871\times 10^{-7}}
\def\ohmtoohm{8.9875517923\times 10^{11}}
\def\btob{1.000000000274\times 10^{-4}}
\def\avtoav{1.000000000274\times 10^{-6}}
\def\wbtowb{1.000000000274\times 10^{-8}}
\def\ytoy{8.9875517923\times 10^{11}}
\def\mutomu{9.99999999726\times 10^{-4}}
\def\magtomag{9.99999999726\times 10^{2}}
\def\htoh{7.95774715242\times 10^{1}}
\def\toverg{9.99999999726\times 10^{3}}

\def\barlcgs{5.29177210903\times 10^{-9}}
\def\barmcgs{9.1093837015\times 10^{-28}}
\def\barrhomcgs{6.1473168257\times 10^{-3}}
\def\barvcgs{2.18769126364\times 10^{8}}
\def\baracgs{9.0442161272\times 10^{24}}
\def\barpcgs{1.99285191410\times 10^{-19}}
\def\baramcgs{1.0545718176462\times 10^{-27}}
\def\barfcgs{8.2387234982\times 10^{-3}}
\def\barucgs{4.3597447222072\times 10^{-11}}
\def\barwcgs{1.8023783420686\times 10^{6}}
\def\barprcgs{2.9421015696\times 10^{14}}
\def\baricgs{1.98571079282\times 10^{7}}
\def\barccgs{4.80320471388\times 10^{-10}}
\def\barrhocgs{3.2413632055\times 10^{15}}
\def\barcurcgs{7.0911019670\times 10^{23}}
\def\barecgs{1.71525554062\times 10^{7}}
\def\barphicgs{9.07674142975\times 10^{-2}}
\def\barcapcgs{5.29177210903\times 10^{-9}}
\def\bardipcgs{2.54174647389\times 10^{-18}}
\def\barpolarcgs{1.71525554062\times 10^{7}}
\def\bardcgs{1.71525554062\times 10^{7}}
\def\barohmcgs{4.57102890439\times 10^{-9}}
\def\barbcgs{2.35051756693\times 10^{9}}
\def\baravcgs{1.24384033025\times 10^{1}}
\def\barwbcgs{6.58211956771\times 10^{-8}}
\def\barycgs{1.10567901732\times 10^{-25}}
\def\barmucgs{1.85480201618\times 10^{-20}}
\def\barmagcgs{1.25168244264\times 10^{5}}
\def\barhcgs{1.25168244264\times 10^{5}}

\def\barmry{1.82187674030\times 10^{-30}}
\def\bartry{4.8377686531714\times 10^{-17}}
\def\barnury{2.0670686667591\times 10^{16}}
\def\barvry{1.09384563182\times 10^{6}}
\def\barary{4.52210806362\times 10^{22}}
\def\barfry{4.11936174912\times 10^{-8}}
\def\barury{2.1798723611036\times 10^{-18}}
\def\barwry{4.505945855171\times 10^{-2}}
\def\barprry{1.47105078482\times 10^{13}}
\def\bariry{2.3418026858671\times 10^{-3}}
\def\barcry{1.1329099625600\times 10^{-19}}
\def\barrhory{7.6452553796\times 10^{11}}
\def\barcurry{8.36272920114\times 10^{17}}
\def\barery{3.63608926151\times 10^{11}}
\def\barphiry{1.9241355740025\times 10^{1}}
\def\bardipry{5.99510134192\times 10^{-30}}
\def\barpolarry{4.0456949184\times 10^{1}}
\def\bardry{3.21946172254}
\def\barohmry{8.2164718044553\times 10^{3}}
\def\barbry{3.3241338227\times 10^{5}}
\def\baravry{1.75905586495\times 10^{-5}}
\def\barwbry{9.3085227643611\times 10^{-16}}
\def\baryry{3.9749389735260\times 10^{-13}}
\def\barmury{6.5577154152\times 10^{-24}}
\def\barmagry{4.42536571420\times 10^{7}}
\def\barhry{3.52159414202\times 10^{6}}

\def\barbg{1.71525554109\times 10^{3}}
\def\baravg{9.0767414322\times 10^{-8}}
\def\barwbg{4.80320471520\times 10^{-18}}
\def\barmug{2.5417464732\times 10^{-21}}
\def\barmagg{1.71525554016\times 10^{10}}
\def\barhg{1.36495698941\times 10^{9}}

\def\cspeedau{1.37035999084\times 10^{2}}
\def\amuau{1.8228884862\times 10^{3}}
\def\ryev{1.3605693122994\times 10^{1}}
\def\hzcmm1{3.33564095198152\times 10^{-11}}
\def\cmm1hz{2.99792458\times 10^{10}}

\def\barifc{1.55689310282\times 10^{3}}
\def\bardmc{3.50506782501\times 10^{-13}}
\def\baralpha{4.57102890439\times 10^{-7}}
\def\baralphap{3.05882401116\times 10^{-10}}
\def\zmtozm{9.99999999726\times 10^{-2}}
\def\alphatoalpha{3.33564095198\times 10^{-9}}
\def\alphaptoalphap{4.19169004390\times 10^{-8}}
\begin{document} 

\author{Andrea Dal Corso \\ (SISSA - Trieste)}
\date{}

\title{
  \includegraphics[width=8cm]{thermo_pw.jpg} \\
  \vspace{3truecm}
  % title
  \Huge \color{dark-blue} Units guide (v.\version)
}

\maketitle

\newpage

\color{dark-blue}
\tableofcontents
\color{black}

\newpage

{\color{dark-blue}\chapter{Introduction}}
\color{black}

These notes discuss the atomic units (a.u.) used in 
electronic structure codes. They are updated with the recent 
(year 2019) changes to the international system (SI).  
The conversion factors written here should be those implemented in
the \qe\ and \tpw\ codes. \\
These notes are part of the \tpw\ package. The complete package is
available at \texttt{https://github.com/dalcorso/thermo\_pw}.

\newpage
{\color{coral}\section{People}}
\color{black}
These notes have been written by Andrea Dal Corso (SISSA - Trieste). \\
Disclaimer: I am not an expert of units. 
These notes reflect what I think about units.
If you think that some formula is wrong, that I misunderstood something, or 
that something can be calculated more simply, please let me know, I would 
like to learn more. 
You can contact me by e-mail: \texttt{dalcorso@sissa.it}. 

\newpage
{\color{coral}\section{Overview}}
\color{black}
Electronic structure codes use atomic units (a.u.). 
In these notes we explain how to obtain the conversion factors 
from a.u. to the international system (SI) units and viceversa. 
The most relevant physical formulas are written both in SI units 
and in a.u.. Moreover, since several books and old literature still 
use the c.g.s. (centimeter-gram-second) system we discuss also how 
to convert from a.u. to c.g.s.-Gaussian units and viceversa.
%and the modified c.g.s.-Gaussian system.

These notes are organized in Sections, one for each physical quantity. 
For each quantity we give its definition in the SI system and then we derive 
the conversion factor from the a.u. to the SI unit, the conversion 
factor from the c.g.s-Gaussian unit to the SI unit, and the conversion 
factor from the a.u. to the c.g.s-Gaussian unit. The text in each Section 
depends on the definitions given in previous Sections. Only in a few cases 
it depends on definitions given in following Sections and in these cases 
we mention explicitly where to find the required definition. Important 
physical formulas are introduced when we have given sufficient information to
convert the formula in different systems. Microscopic and macroscopic 
Maxwell's equations are also summarized in separate Sections. \\
Each system has a different color. In black the SI, in blue the a.u., in 
orange the c.g.s.-Gaussian system, 
%in violet the modified c.g.s.-Gaussian system (when different from the
%c.g.s.-Gaussian system) 
and in green the conversion factors from a.u. to the c.g.s.-Gaussian units. 
Comments of interest not belonging to any system in particular are given 
in red. We indicate the numerical value of a given quantity in the SI 
with a tilde. The units in a.u. are indicated with a bar, while the units 
in the c.g.s.-Gaussian system are indicated with a bar and the subscript 
${\rm cgs}$. When the c.g.s.-Gaussian unit has an accepted name we use it 
interchangeably with the generic name. Only Hartree a.u. are described 
in the main text since these are the most common microscopic units. 
\qe\ uses Rydberg a.u. that can be easily derived from the Hartree
a.u.. We describe Rydberg a.u. in Appendix A (in purple). In the $ab-initio$
literature modified a.u. have been introduced in which the 
electromagnetic equations look like those of the c.g.s-Gaussian system. 
This requires some modifications to the definitions given in the main 
text and we discuss these units in Appendix B (in steelblue).

It is useful to recall a few preliminary facts needed in the rest of 
these notes. \\
A few experimental quantities have fixed values in the SI. Among these:
\\
The Planck constant 
\begin{equation}
h=\hplanck\ {\rm J}\cdot {\rm s}, 
\label{hplanck}
\end{equation}
the speed of light 
\begin{equation}
c=\cspeed\ {\rm m}/{\rm s},
\label{cspeed}
\end{equation}
 and the electron charge
\begin{equation}
e=\e\ {\rm C}. 
\label{echarge}
\end{equation}
Two experimental quantities are known with high accuracy:
\\
The Rydberg constant, measured spectroscopically from the frequencies of
Hydrogen and Deuterium absorption and emission lines:
\begin{equation}
R_\infty=
\rydberg\ 1/{\rm m}
\label{rydberg}
\end{equation}
is known with a relative error of $1.9\times 10^{-12}$ \\
and the fine structure constant, measured from the anomaly of the
electron magnetic moment,
\begin{equation}
\alpha=
\alphaf,
\label{alpha}
\end{equation}
is known with a relative error of $1.6\times 10^{-10}$.

The Rydberg constant can be written in terms of the fine structure
constant $\alpha$ and the electron mass $m_e$ or in terms of $\alpha$ and
of the Bohr radius $a_B$:
\begin{equation}
R_\infty={\alpha^2 m_e c \over 2 h}={\alpha \over 4 \pi a_B}.
\end{equation}
Using these equations we can calculate the electron mass as:
\begin{equation}
m_e={2 h R_\infty\over \alpha^2 c}=
\me\ {\rm kg}
\end{equation} 
and $a_B$ as:
\begin{equation}
a_B={\alpha\over 4\pi R_\infty}=
\abohr\ {\rm m}.
\end{equation}
\\
$\alpha$ is a pure number equal to:
\begin{equation}
\alpha={e^2 \over 4 \pi \epsilon_0 \hbar c}= {\mu_0 c e^2 \over 2 h}, 
\label{alphadef}
\end{equation}
where $\hbar=h/2\pi$, $\epsilon_0$ the vacuum electric
permittivity and $\mu_0$ is the vacuum magnetic permeability. Hence
the Bohr radius can be rewritten as:
\begin{equation}
a_B={\hbar \over \alpha m_e c}=
{\hbar^2 4 \pi \epsilon_0 \over m_e e^2}.
\end{equation}
Eq.~\ref{alphadef} allows the calculation of $\mu_0$ as:
\begin{equation}
\mu_0={2 h \alpha \over e^2 c}=\muzero\ {{\rm N}\over {\rm A}^2}.
\label{mu0}
\end{equation}
$\epsilon_0$ is obtained from the relation:
\begin{equation}
\epsilon_0={1 \over \mu_0 c^2}=\epsilonzero\ 
{{\rm C}^2\over {\rm N}\cdot {\rm m}^2}.
\label{epsilon0}
\end{equation}
It is also useful to define the hartree energy:
\begin{equation}
E_h= {e^2\over 4\pi\epsilon_0 a_B}=2 h c R_\infty = \baru\ {\rm J},
\label{eh}
\end{equation}
and the Bohr magneton:
\begin{equation}
\mu_B = {\hbar e \over 2 m_e} = \bohrmag\ {\rm J}/{\rm T}.
\label{bohrm}
\end{equation}

\newpage

{\color{dark-blue}\chapter{Mechanical quantities}}

{\color{coral}\section{Time}}
\color{black}
In the SI, the unit of time is the second 
(symbol ${\rm s}$), defined requiring that the frequency of a particular 
line of the $^{133}$Cs atom is exactly $9192631770\ {\rm s}^{-1}$.
\\

{\color{web-blue} In a.u. the unit of time (symbol ${\rm \bar t}$) is 
defined requiring that the numerical value of $\hbar$ is $1$.
The conversion factor with the SI unit is obtained recalling that in the 
SI the Planck constant 
is $\hbar= \tilde \hbar\ {\rm J}\cdot {\rm s}=\tilde \hbar\ {\rm kg}\cdot {\rm m}^2 / 
{\rm s}$. 
Therefore in a.u. we have: 
\begin{equation}
\hbar = {{\rm \bar m} \cdot {\rm \bar l}^2 \over {\rm \bar t}},
\end{equation}
where ${\rm \bar l}$ is the unit of length and ${\rm \bar m}$ the unit of mass.
As we discuss below ${\rm \bar l}=a_B$ and ${\rm \bar m}=m_e$, therefore
\begin{equation}
{\rm \bar t} = {m_e a_B^2 \over \hbar} = 
{\hbar 4 \pi \epsilon_0 a_B \over e^2}
={\hbar \over E_h}= {\hbarf\ {\rm J}\cdot {\rm s} \over \baru\ {\rm J}}=
\bart\ {\rm s}
\end{equation}
\\
}

{\color{orange} In the c.g.s. system the unit of time 
is the second (symbol ${\rm s}$), defined as in the SI. We have ${\rm \bar t_{cgs}}={\rm s}$.
\\
}

{\color{green} The conversion factor between the a.u. and the c.g.s. unit is:
${\rm \bar t}=\bart\ {\rm s}$.
}

\newpage

{\color{coral}\section{Length}}
\color{black}
In the SI, the unit of length is the metre (symbol ${\rm m}$) defined
as the length of the path traveled by light during the time interval
of $1/299792458\ {\rm s}$.
\\

{\color{web-blue} In a.u. the unit of the length (symbol ${\rm \bar l}$) is defined
requiring that the Bohr radius $a_B= {\rm \bar l}$. The conversion factor with the
SI unit is ${\rm \bar l} = \barl\ {\rm m}$.}
\\

{\color{orange} In the c.g.s. system the unit of length is the centimetre 
(symbol ${\rm cm}$) defined as ${\rm cm} =\cmtom\ {\rm m}$. We have ${\rm \bar l_{cgs}}=10^{-2}\ {\rm m}$.
\\
}

{\color{green} The conversion factor between the a.u. and the c.g.s. unit is:
${\rm \bar l}=\barlcgs\ {\rm cm}$.
\\
}

{\color{red} A common unit of length is the angstrom (symbol \AA).
We have \AA$=10^{-10}\ {\rm m}$.
}

\newpage

{\color{coral}\section{Mass}}
\color{black}
In the SI the unit of mass is the kilogram (symbol ${\rm kg}$) defined
requiring that the Planck constant $h$ has the value
$\hplanck$ ${\rm kg} \cdot {\rm m}^2 / {\rm s}$. 
\\

{\color{web-blue} In a.u. the unit of mass (symbol ${\rm \bar m}$) is defined
requiring that the mass of the electron $m_e= {\rm \bar m}$. The conversion
factor with the SI unit is ${\rm \bar m} = \me\ {\rm kg}$.
}
\\

{\color{orange} In the c.g.s. system the unit of mass is the gram 
(symbol ${\rm g}$) defined as ${\rm g} =\gtokg\ {\rm kg}$. 
\\
}

{\color{green} The conversion factor between the a.u. and the c.g.s. unit is
${\rm \bar m}=\barmcgs\ {\rm g}$.
\\
}

{\color{red} Atomic weights are usually expressed in atomic mass units
(${\rm a.m.u.}=$ \\ $\amu\ {\rm kg}$). This quantity can be converted in a.u. as:
\begin{equation}
{\rm a.m.u.} = {\amu\ {\rm kg} \over \barm\ {\rm kg} }\ {\rm \bar m}=\amuau\ {\rm \bar m}
\end{equation}
}

\newpage
{\color{coral}\section{Mass density}}
\color{black}
In the SI the unit of mass density is derived from its definition:

\begin{tcolorbox}
\begin{equation}
\rho_m= {d m \over d V},
\end{equation}
\end{tcolorbox}

and it is ${\rm kg}/{\rm m}^3$.
\\

{\color{web-blue} In a.u. the unit of mass density 
(symbol ${\rm \bar \rho_m}$) is derived from
its definition: ${\rm \bar \rho_m}= {{\rm \bar m} \over {\rm \bar l}^3}$.
The conversion factor with the SI unit is:
\begin{equation}
{\rm \bar \rho_m} = {m_e \over a_B^3}= {\me\ {\rm kg} \over (\abohr\ {\rm m})^3}=
\barrhom\ {\rm kg}/{\rm m}^3.
\end{equation}
}
\\

{\color{orange} In the c.g.s. system the unit of mass density is derived from
its definition: ${{\rm g}/{\rm cm}^3}$. The conversion factor with
the SI unit is ${{\rm g}/{\rm cm}^3}=10^3\ {\rm kg}/{\rm m}^3$.
}
\\

{\color{green} The conversion factor between the a.u. and the c.g.s. unit is
${\rm \bar \rho_m}=\barrhomcgs\ {\rm g}/{\rm cm}^3$.
}

\newpage
{\color{coral}\section{Frequency}}
\color{black}

In the SI the unit of frequency (symbol ${\rm Hz}$) is the inverse
of the unit of time: ${\rm Hz}=1/{\rm s}$. This unit is called hertz.
The angular frequency is defined as $\omega = 2 \pi \nu$ where 
$\nu$ is the frequency. Its units are ${\rm radiant}/{\rm s}$.
\\

{\color{web-blue} In a.u. the unit of frequency (symbol ${\rm \bar \nu}$) is defined
in a similar way ${\rm \bar \nu}= 1/\bar t$. The conversion factor with the 
SI unit is:
\begin{equation}
{\rm \bar \nu} = {1 \over \bart\ {\rm s}}=\barnu\ {\rm Hz}.
\end{equation}
}
\\

{\color{orange} In the c.g.s. system the unit of frequency is the hertz 
defined as in the SI.
\\
}

{\color{green} The conversion factor between the a.u. and the c.g.s. unit is
${\rm \bar \nu}=\barnu\ {\rm Hz}$.
\\
}

{\color{red}
A commonly used unit of frequency is the wavenumber, that is the number of 
light waves with frequency $\nu$ per ${\rm cm}$. If the wavelength $\lambda$ is
given in ${\rm cm}$, the wavenumber is $\bar \nu={1\over \lambda}$ and its units
are ${\rm cm}^{-1}$. Since $\lambda={c\over \nu}$ the conversion factor
from ${\rm Hz}$ to ${\rm cm}^{-1}$ is:
\begin{equation}
{1\over 10^2 c}=\hzcmm1\ {\rm cm}^{-1}/{\rm Hz},
\end{equation}
while the conversion factor from ${\rm cm}^{-1}$ to ${\rm Hz}$ is:
\begin{equation}
10^2 c=\cmm1hz\ {\rm Hz}/{\rm cm}^{-1}
\end{equation}
}

\newpage
{\color{coral}\section{Speed}}
\color{black}
In the SI the unit of speed is derived from its definition.
For instance, the speed (${\bf v}$) of a particle 
whose position as a function of time is ${\bf r}(t)$, is:

\begin{tcolorbox}
\begin{equation}
{\bf v}={d{\bf r} \over dt},
\end{equation} 
\end{tcolorbox}

so the unit of speed is ${\rm m}/{\rm s}$.
\\

{\color{web-blue} In a.u. the unit of speed (${\rm \bar v}$) is derived 
from its definition ${\rm \bar v} = {\rm \bar l} / {\rm \bar t}$. The conversion factor
with the SI unit is:
\begin{align}
{\rm \bar v} ={a_B \hbar \over m_e a_B^2}={\hbar \over m_e a_B}=\alpha c&=
 {\alphaf\  \cspeed\ {\rm m}/{\rm s}}\nonumber \\ 
&=\barv\ {\rm m}/{\rm s}.
\end{align}
Using the definitions of ${\rm \bar l}$ and ${\rm \bar t}$ we can also write:
\begin{equation}
{\rm \bar v}={a_B E_h \over \hbar}.
\end{equation}
\\
Note that since ${\rm \bar v}= \alpha c$ we can also write 
$c={1\over \alpha} {\rm \bar v}$ meaning that the speed of light in a.u. 
$c_{a.u.}={1\over\alpha}=\cspeedau$.
}
\\

{\color{orange} In the c.g.s. system the unit of speed is the 
${\rm \bar v_{cgs}}={\rm cm}/{\rm s} = \cmtom\ {\rm m}/{\rm s}$.
\\
}

{\color{green} The conversion factor between the a.u. and the c.g.s. unit is:
${\rm \bar v}=\barvcgs\ {\rm cm}/{\rm s}$.
\\
}

\newpage
{\color{coral}\section{Acceleration}}
\color{black}
In the SI the unit of acceleration is derived from its  
definition. For instance, the acceleration (${\bf a}$) of a particle whose
position as a function of time is ${\bf r}(t)$, is: 

\begin{tcolorbox}
\begin{equation}
{\bf a}={d^2{\bf r} \over dt^2},
\end{equation} 
\end{tcolorbox}

so the unit of acceleration is ${\rm m}/{\rm s}^2$.
\\

{\color{web-blue} In a.u. the unit of acceleration (${\rm \bar a}$) is derived 
from its definition ${\rm \bar a} = {\rm \bar l} / {\rm \bar t}^2$. The conversion factor
with the SI unit is:
\begin{align}
{\rm \bar a} = {\bar v \over \bar t}={E_h \alpha c \over \hbar}&= 
{\baru\ {\rm J}\ \alphaf\ \cspeed\ {\rm m}/{\rm s} \over \hbarf\ {\rm J}\cdot {\rm s}}\nonumber \\
&= \bara\ {\rm m}/{\rm s}^2.
\end{align}
\\
}

{\color{orange} In the c.g.s. system the unit of acceleration is the ${\rm \bar a_{cgs}}={\rm cm}/{\rm s}^2 = \cmtom\ {\rm m}/{\rm s}^2$.
\\
}

{\color{green} The conversion factor between the a.u. and the c.g.s. unit is:
${\rm \bar a}=\baracgs\ {\rm cm}/{\rm s}^2$.
\\
}

\newpage
{\color{coral}\section{Momentum}}
\color{black}
In the SI the unit of momentum is derived from its definition.
For instance, the momentum (${\bf p}$) of a particle of mass ${\rm m}$ whose
position as a function of time is ${\bf r}(t)$, is:

\begin{tcolorbox}
\begin{equation}
{\bf p}=m {d{\bf r} \over dt},
\end{equation} 
\end{tcolorbox}

so the unit of momentum is ${\rm kg}\cdot {\rm m}/{\rm s}$.
\\

{\color{web-blue} In a.u. the unit of momentum (${\rm \bar p}$) is derived 
from its definition ${\rm \bar p} = {\rm \bar m} \cdot {\rm \bar v}$. 
The conversion factor with the SI unit is:
\begin{equation}
{\rm \bar p} = {m_e a_B \over {\rm \bar t}}= {\hbar \over a_B} =
{\hbarf\ {\rm kg}\cdot {\rm m}^2/{\rm s} \over \barl\ {\rm m}}=\barp\ {\rm kg}
\cdot {\rm m}/{\rm s}.
\end{equation}
\\ }

{\color{orange} In the c.g.s. system the unit of momentum is 
${\rm \bar p_{cgs}}={\rm g}\cdot {\rm cm}/{\rm s} = \ptop\ {\rm kg}\cdot {\rm m}/{\rm s}$.
\\
}

{\color{green} The conversion factor between the a.u. and the c.g.s. unit is:
${\rm \bar p}=\barpcgs\ {\rm g}\cdot {\rm cm}/{\rm s}$.
\\
}

\newpage
{\color{coral}\section{Angular momentum}}
\color{black}
In the SI the unit of angular momentum is derived from its
definition. For instance, the angular momentum (${\bf L}$) 
of a particle at position ${\bf r}$ and with momentum ${\bf p}$ is:

\begin{tcolorbox}
\begin{equation}
{\bf L}= {{\bf r} \times {\bf p}},
\end{equation} 
\end{tcolorbox}

so the unit of angular momentum is ${\rm kg}\cdot {\rm m}^2/{\rm s}$.
\\

{\color{web-blue} In a.u. the unit of angular momentum (${\rm \bar L}$) is derived 
from its definition
${\rm \bar L} = {{\rm \bar m} \cdot {\rm \bar l}^2 \over {\rm \bar t}}$. 
The conversion factor
with the SI unit is:
\begin{equation}
{\rm \bar L}={m_e a_B^2 \hbar \over m_e a_B^2}=\hbar=
\hbarf\ {\rm kg}\cdot {\rm m}^2/{\rm s}.
\end{equation}
\\
}

{\color{orange} In the c.g.s. system the unit of angular momentum is 
${\rm \bar L_{cgs}}={\rm g}\cdot {\rm cm}^2/{\rm s} = \ltol\ {\rm kg}\cdot {\rm m}^2/{\rm s}$.
\\
}

{\color{green} The conversion factor between the a.u. and the c.g.s. unit is:
${\rm \bar L}=\baramcgs\ {\rm g}\cdot {\rm cm}^2/{\rm s}$.
\\
}

\newpage
{\color{coral}\section{Force}}
\color{black}
In the SI the unit of force (symbol ${\rm N}$) is derived from  
the Newton equation:

\begin{tcolorbox}
\begin{equation}
{\bf F}= m {\bf a},
\end{equation} 
\end{tcolorbox}

so the unit of force is ${\rm N}={\rm kg}\cdot {\rm m}/{\rm s}^2$. 
This unit is called newton.
\\

{\color{web-blue} In a.u. the unit of force (${\rm \bar f}$) is derived 
from the same equation so we have ${\rm \bar f} ={\rm \bar m} \cdot {\rm \bar a}$. The
conversion factor with the SI unit is: 
\begin{align}
{\rm \bar f} &= {m_e a_B \hbar E_h \over m_e a_B^2 \hbar }= 
{E_h\over a_B}\nonumber={\baru {\rm J} \over \barl {\rm m}}\nonumber \\
&=\barf\ {\rm N}. 
\end{align}
So in a.u. the unit of force is hartree/bohr.
}
\\

{\color{orange} In the c.g.s. system the unit of force is the 
${\rm \bar f_{cgs}}={\rm g}\cdot {\rm cm}/{\rm s}^2 = \ftof\ {\rm kg}\cdot {\rm m}/{\rm s}^2$. This unit is 
called dyne.
\\
}

{\color{green} The conversion factor between the a.u. and the c.g.s. unit is:
${\rm \bar f}=\barfcgs\ {\rm dyne}$.
\\
}
\newpage

{\color{coral}\section{Energy}}
\color{black}
In the SI the unit of energy (symbol ${\rm J}$) is derived from the 
definition of work: 

\begin{tcolorbox}
\begin{equation}
U= \int {\bf F} \cdot d{\bf r}
\end{equation} 
\end{tcolorbox}

so the unit of energy is ${\rm J}={\rm kg}\cdot {\rm m}^2/{\rm s}^2={\rm N}\cdot {\rm m}$. This unit is 
called joule.
\\

{\color{web-blue} In a.u. the unit of energy (${\rm \bar U}$) is derived 
from the definition of work so ${\rm \bar U} ={\rm \bar f} \cdot {\rm \bar l}$. The 
conversion factor with the SI unit is: 
\begin{equation}
{\rm \bar U} = {E_h\over a_B} a_B=E_h=\baru\ {\rm J}, 
\end{equation}
so in a.u. the unit of the energy is the hartree.
}
\\

{\color{orange} In the c.g.s. system the unit of energy (symbol ${\rm erg}$)
is ${\rm \bar U_{cgs}}={\rm erg}={\rm dyne}\cdot {\rm cm} = \utou\ {\rm N}\cdot {\rm m}$. This unit is 
called erg.
\\
}

{\color{green} The conversion factor between the a.u. and the c.g.s. unit is:
${\rm \bar U}=\barucgs\ {\rm erg}$.
\\
}

\newpage
{\color{coral}\section{Power}}
\color{black}
In the SI the unit of power (symbol ${\rm W}$) is derived from its
definition as the work per unit time: 

\begin{tcolorbox}
\begin{equation}
P= {d U \over d t},
\end{equation} 
\end{tcolorbox}

so the unit of power is ${\rm W}={\rm J}/{\rm s}$. This unit is called watt.
\\

{\color{web-blue} In a.u. the unit of power (${\rm \bar W}$) is derived 
from its definition ${\rm \bar W} ={{\rm \bar U} \over {\rm \bar t}}$. The
conversion factor with the SI unit is: 
\begin{equation}
{\rm \bar W} = {\baru\ {\rm J}\over \bart\ {\rm s}}=\barw\ {\rm W}. 
\end{equation}
Using the definition of ${\rm \bar t}$ and ${\rm \bar U}$ we can also write:
\begin{equation}
{\rm \bar W} = {E_h^2 \over \hbar}.
\end{equation}
}
\\

{\color{orange} In the c.g.s. system the unit of power is 
${\rm \bar W_{cgs}} = {\rm erg}/{\rm s} = \utou\ {\rm W}$. 
\\
}

{\color{green} The conversion factor between the a.u. and the c.g.s. unit is:
${\rm \bar W}=\barwcgs\ {\rm erg}/{\rm s}$.
}

\newpage
{\color{coral}\section{Pressure}}
\color{black}
In the SI the unit of pressure (or stress) (symbol ${\rm Pa}$) is derived 
from its definition as a force per unit area, so 
the unit of pressure is ${\rm N}/{\rm m}^2={\rm Pa}$. This unit is called 
pascal.
\\

{\color{web-blue} In a.u. the unit of pressure (${\rm \bar \sigma}$) is derived from 
its definition so ${\rm \bar \sigma}= {{\rm \bar f} \over {\rm \bar l}^2}$. The conversion
factor with the SI unit is:
\begin{equation}
{\rm \bar \sigma}= {\barf\ {\rm N} \over (\barl\ {\rm m} )^2}=\barpr\ {\rm Pa}
\end{equation}
Note also that ${\rm \bar f} = {E_h \over a_B}$ so 
${\rm \bar \sigma} = {E_h \over a_B^3}$.
}

{\color{orange} In the c.g.s. system the unit of pressure (symbol ${\rm Ba}$)
is the ${\rm \bar \sigma_{cgs}}={\rm Ba}={\rm dyne}/{\rm cm}^2 = \prtopr\ {\rm Pa}$. This unit is 
called barye. 
\\
}

{\color{green} The conversion factor between the a.u. and the c.g.s. unit is:
${\rm \bar \sigma}=\barprcgs\ {\rm Ba}$.
\\
}

{\color{red} Other common units of pressure are:
${\rm bar}=10^5\ {\rm Pa}$, atmosphere (${\rm atm}=1.01325\ {\rm bar}$), ${\rm torr}= {1\over 760}\ {\rm bar}$, 
millimeters of mercury (${\rm mmHg}=1\ {\rm torr}$).
}

\newpage
{\color{coral}\section{Temperature}}
\color{black}
In the SI the unit of temperature is the kelvin (symbol ${\rm K}$) defined
so that the Boltzmann constant is $\kb {\rm J} / {\rm K}$.
\\

{\color{web-blue} In a.u. the unit of temperature (symbol ${\rm K}$) is the 
kelvin as in the SI.}
\\

{\color{orange} In the c.g.s. system the unit of temperature 
(symbol ${\rm K}$) 
is the kelvin as in the SI.}
\\

{\color{red} In the SI the Avogadro number is exact and given by:
\begin{equation}
N_A=\avonum,
\end{equation}
and the universal gas constant is 
\begin{equation}
R=k_B N_A = \kb\ {{\rm J}\over {\rm K}} \times \avonum = \rgas\ {{\rm J}
\over {\rm K}}.
\end{equation}
\\

The heat necessary to increase the temperature of water from $16.5$ 
$^\circ{\rm C}$
to $17.5$ $^\circ{\rm C}$ is known as (thermochemical) calorie 
(symbol ${\rm cal}$) and 
is equal to $4.184\ {\rm J}$. This is the definition that we use in \tpw. 
There is also another definition of calorie called international calorie and
equal to $4.1868\ {\rm J}$, but we do not use it.
}

\newpage

{\color{dark-blue}\chapter{Electromagnetic quantities}

{\color{coral}\section{Current}}
\color{black}
In the SI the unit of current is the ampere (symbol ${\rm A}$) defined
requiring that the charge of the electron is $e=\e$ ${\rm A}\cdot {\rm s}$.
The quantity ${\rm A}\cdot {\rm s}$ is called coulomb (symbol ${\rm C}$).
\\

{\color{web-blue} In a.u. the unit of the current (symbol ${\rm \bar I}$) is  
derived from its definition:
\begin{equation}
I={dq \over dt}, 
\end{equation}
where $dq$ is the charge that passes through a cross section of the conductor
in the time $dt$, so ${\rm \bar I} = {\rm \bar C} / {\rm \bar t}$ where 
${\rm \bar C}$ is the 
unit of charge (equal to $e$, see below). The conversion factor with the SI unit is:
\begin{equation}
{\rm \bar I} = {e E_h \over \hbar}={\barc\ {\rm C}\ \baru {\rm J} \over 
\hbarf\  {\rm J}\cdot {\rm s}} = \bari {\rm A}.
\end{equation}
}
\\

{\color{orange} In the c.g.s.-Gaussian system the unit of current 
(symbol ${\rm statA}$) is defined from Amp\`ere law. The force per unit length 
between two parallel wires that carry a current $I$ and $I'$ is:
\begin{equation}
{F\over l} = {2\over c^2} {I I'\over r},
\label{force_wire}
\end{equation}
therefore ${\rm \bar I_{cgs}}={\rm statA}=\sqrt{\rm dyne}\ {\rm cm}/{\rm s}$. This unit is called 
statampere or 
electrostatic unit (${\rm esu}$) of current.

In order to convert between ${\rm statA}$ and ${\rm A}$, we can write 
${\rm A}= \mathcal{K}\ {\rm statA}$. 
When the currents in the two wires are $I\ {\rm statA}={I\over \mathcal{K}} {\rm A}$ and 
$I'\ {\rm statA}={I'\over \mathcal{K}} {\rm A}$ and their distance
is $r\ {\rm cm}=r\ 10^{-2}\ {\rm m}$ the force per metre is in 
${\rm N}/{\rm m}=
{10^5\over 10^2} {\rm dyne}/{\rm cm}$
\begin{equation}
F=2{\tilde \mu_0\over 4\pi} {10^3\over 10^{-2} \mathcal{K}^2}\ {I\ I' \over r} {\rm dyne/cm},
\end{equation}
where $\tilde \mu_0$ is the numerical value of $\mu_0$ in SI units
(see Eq.~\ref{mu0}). Comparing this equation with Eq.~\ref{force_wire} we
obtain:
\begin{equation}
{\tilde \mu_0 \over 4 \pi} {10^5\over \mathcal{K}^2} = 
{1\over \tilde c_{cgs}^2}, 
\end{equation}
where $\tilde c_{cgs}$ is the numerical value of the speed of light in
c.g.s. units. We have $\tilde c_{cgs}=10^2 \tilde c$ where $\tilde c$ is
the numerical value of the speed of light in SI units. From this
equation we obtain:
\begin{equation}
\mathcal{K} = \tilde c_{cgs} \sqrt {10^5 \tilde \mu_0 \over 4 \pi} =
{\tilde c_{cgs}\over \tilde c} 
\sqrt {10^5 \over 4 \pi \tilde \epsilon_0}=\sqrt {10^9 \over 4 \pi 
\tilde \epsilon_0}=\kappa,
\end{equation}
where $\tilde \epsilon_0$ is the numerical value of the vacuum permittivity
in SI units (Eq.~\ref{epsilon0}).

In the old SI units ${\tilde \mu_0 \over 4 \pi} = 10^{-7}$, so 
$\mathcal{K}=10^{-1} \tilde c_{cgs}= 10\ \tilde c$.
In the new SI we have:
\begin{equation}
{\mathcal{K}\over 10 \tilde c}= \kappadiecic...
\end{equation}
\\
}

%{\color{violet} There exists also a modified c.g.s. Gaussian system, 
%where the unit of the current (symbol $abA$) is defined 
%from:
%\begin{equation}
%{F\over l} = 2 {I I'\over r}.
%\end{equation}
%Therefore $abA=\sqrt{dyne}$. This unit is called abampere or 
%electromagnetic unit ($emu$) of current or biot.
%
%In order to compare $abA$ and $A$ we can write $abA= {1\over \mathcal{K}_A}\ A$ 
%and reasoning
%as before we have:
%\begin{equation}
%{\tilde \mu_0 \over 2 \pi}\ {10^5\over \mathcal{K}_A^2} = 1
%\end{equation}
%or
%\begin{equation}
%\mathcal{K}_A = \sqrt {10^5 \tilde \mu_0 \over 4 \pi} ={1\over \tilde c} 
%\sqrt {10^5 \over 4 \pi \tilde \epsilon_0}={\mathcal{K}\over \tilde c_{cgs}} 
%=\kappaa.
%\end{equation}
%\\
%In the old SI units ${\tilde \mu_0 \over 4 \pi} = 10^{-7}$, so 
%$\mathcal{K}_A=10^{-1}$.
%}
%\\

{\color{green} The conversion factor between the a.u. and the c.g.s.-Gaussian 
unit is: ${\rm \bar I}=\bari\ \mathcal{K}\ {\rm statA}=\baricgs\ {\rm statA}$. \\
%The conversion factor between a.u. and modified c.g.s.-Gaussian units is:
%$\bar I=\bari\ \mathcal{K}_A abA=\baricgsm\ abA$.
}
\\

\newpage
{\color{coral}\section{Charge}}
\color{black}
In the SI the unit of the charge (symbol ${\rm C}$), derived from the 
definition of the current:

\begin{tcolorbox}
\begin{equation}
I={dq \over dt},
\label{current}
\end{equation}
\end{tcolorbox}

is the charge that passes in one second through the section of a conductor
in which the current is $1 {\rm A}$. This unit is called coulomb.

In the SI the Coulomb force between two charges $q$ and $q'$ at
distance ${\bf r}$ is:

\begin{tcolorbox}
\begin{equation}
{\bf F} = {1\over 4 \pi \epsilon_0} {q\ q' \over r^3}\ {\bf r},
\label{coulomb}
\end{equation}
\end{tcolorbox}

where $r=|{\bf r}|$.
\\

{\color{web-blue} In a.u. the unit of the charge (symbol ${\rm \bar C}$) is
defined requiring that the electron has charge $e={\rm \bar C}$. The 
conversion 
factor between a.u. and the SI unit is: 
\begin{equation}
{\rm \bar C}=\barc\ {\rm C}.
\end{equation}

With this information we can derive the form of the Coulomb law in a.u..
A charge $q\ {\rm \bar C}$ at a distance $r\ {\rm \bar l}$ from a
charge $q'\ {\rm \bar C}$ will fill a force $F {\rm \bar f}$ that can be 
calculated using Eq.~\ref{coulomb}:
\begin{equation}
{\bf F} = {1\over 4 \pi \epsilon_0}\ {{\rm \bar C}^2 \over {\rm \bar l}^2 {\rm \bar f}} 
{q\ q' \over r^3}\ {\bf r},
\end{equation}
but since
\begin{equation}
{1\over 4 \pi \epsilon_0}\ {{\rm \bar C}^2 \over {\rm \bar l}^2{\rm \bar f}} = 
{E_h \over a_B {\rm \bar f}}=1
\end{equation}
the Coulomb law in a.u. is:
\begin{equation}
{\bf F} = {q\ q' \over r^3}\ {\bf r}. 
\end{equation}
}
\\

{\color{orange} In the c.g.s. system the unit of charge (symbol ${\rm statC}$)
is defined from the Coulomb law:
\begin{equation}
{\bf F} = {q\ q'\over r^3}\ {\bf r},
\label{coulombcgs}
\end{equation}
Therefore the charge unit is ${\rm \bar C_{cgs}}={\rm statC}=\sqrt{{\rm dyne}}\cdot {\rm cm}=
{\rm statA}\cdot {\rm s}$. This unit is called statcoulomb 
or electrostatic unit (${\rm esu}$) of charge or franklin.

The conversion factor between ${\rm statC}$ and ${\rm C}$ can be found by 
writing
${\rm statC}={1\over \mathcal{K}} {\rm C}$ and considering two charges of 
$q\ {\rm statC}$ and 
$q'\ {\rm statC}$ respectively, at a distance of $r\ {\rm cm}$. Since the two charges are
of ${q\over \mathcal{K}}\ {\rm C}$ and ${q'\over \mathcal{K}}\ {\rm C}$ at the distance of $r\ 10^{-2}\ {\rm m}$,
we can use the Coulomb law in SI units to find 
the force (in ${\rm N}=10^5 {\rm dyne}$) acting between them: 
\begin{equation}
{\bf F}={1\over 4 \pi \tilde \epsilon_0} {10^5\over 10^{-4} \mathcal{K}^2} 
{q\ q'\over r^3}\ {\bf r}\ {\rm dyne}
\end{equation}
and comparing this equation with Eq.~\ref{coulombcgs} we find:
\begin{equation}
\mathcal{K}= \sqrt{10^9 \over 4 \pi \tilde \epsilon_0}= \kappa. 
\end{equation}
\\
Using this conversion factor we can write the equation that defines 
the current. When a charge $dq\ {\rm statC}$, or ${dq\over \mathcal{K}}\ 
{\rm C}$ passes in $dt\ {\rm s}$
through a section of a conductor, the current (due to Eq.~\ref{current})
is $I={1\over \mathcal{K}}\ {dq\over dt}\ {\rm A}= {dq\over dt}\ {\rm statA}$,
so in the 
c.g.s.-Gaussian system Eq.~\ref{current} still holds.   
}
\\

%{\color{violet} In the modified c.g.s.-Gaussian system the unit of the
%charge is the $statC$. Using the conversion factor between $statC$ and $C$
%we can find the equation that defines the current in this system.
%When a charge of $dq\ statC={dq\over \mathcal{K}}\ C$ 
%passes in a time $dt\ s$ through a section of a conductor,
%the current is (due to Eq.~\ref{current}) 
%$I={1\over \mathcal{K}}\ {dq\over dt}\ A={\mathcal{K}_A \over \mathcal{K}}\ {dq\over dt}\ abA= 
%{1\over c_{cgs}}\ {dq\over dt}\ abA$. So in these units the relationship
%between current and charge is:
%\begin{equation}
%I={1\over c} {dq\over dt}.
%\end{equation}
%}
%\\

{\color{green} The conversion factor between the a.u. and the 
c.g.s.-Gaussian unit is: ${\rm \bar C}=\barc\ \mathcal{K}\ {\rm statC}=
\barccgs\ {\rm statC}$. \\
}

\newpage
{\color{coral}\section{Charge density}}
\color{black}
In the SI the unit of the charge density is derived from its
definition:

\begin{tcolorbox}
\begin{equation}
\rho={dq \over dV},
\end{equation}
\end{tcolorbox}

so the unit of the charge density is ${\rm C}/{\rm m}^3$.
\\

{\color{web-blue} In a.u. the unit of charge density (symbol ${\rm \bar \rho}$) 
is derived its definition ${\rm \bar \rho} ={{\rm \bar C} \over {\rm \bar l}^3}$. The
conversion factor with the SI unit is: 
\begin{equation}
{\rm \bar \rho} = {e\over a_B^3} =
{\barc\ {\rm C}\over (\barl\ {\rm m})^3}=\barrho\ {\rm C}/{\rm m}^3. 
\end{equation}
\\
}

{\color{orange} In the c.g.s. system the unit of charge density  
is derived from its definition and it is
${\rm \bar \rho_{cgs}}={\rm statC}/{\rm cm}^3 = {1 \over \mathcal{K}\ 10^{-6}}\ {\rm C}/{\rm m}^3=\rhotorho\ {\rm C}/{\rm m}^3$.
\\
}

{\color{green} The conversion factor between the a.u. and the 
c.g.s.-Gaussian unit is: ${\rm \bar \rho} = \barrhocgs\ {\rm statC}/{\rm cm}^3$. 
\\
}

\newpage
{\color{coral}\section{Current density}}
\color{black}

In the SI the unit of the current density is derived from 
its definition:

\begin{tcolorbox}
\begin{equation}
I=\int {\bf J}\cdot \hat {\bf n}\ dS,
\end{equation}
\end{tcolorbox}

so the unit of the current density is ${\rm A}/{\rm m}^2$.
\\
In SI units the continuity equation is:

\begin{tcolorbox}
\begin{equation}
{\partial \rho \over \partial t}=-\nabla \cdot {\bf J}.
\label{continuitysi}
\end{equation}
\end{tcolorbox}

{\color{web-blue} In a.u. the unit of the current density (symbol ${\rm \bar J}$) 
is derived 
from the same equation so we have ${\rm \bar J} ={{\rm \bar I} \over {\rm \bar l}^2}$. The
conversion factor to SI units is: 
\begin{equation}
{\rm \bar J} = {\bari\ {\rm A}\over (\barl\ {\rm m})^2}=\barcur\ {\rm A}/{\rm m}^2. 
\end{equation}
\\
Inserting the expressions of ${\rm \bar I}$ and ${\rm \bar l}$ we can also write:
\begin{equation}
{\rm \bar J} = {e E_h \over \hbar a_B^2}. 
\end{equation}
\\
The continuity equation can be written as:
\begin{equation}
{\partial \rho \over \partial t}=-{{\rm \bar t \bar J} \over {\rm \bar \rho \bar l}} 
\nabla \cdot {\bf J}.
\label{continuitygen}
\end{equation}
Since ${{\rm \bar t \bar J} \over {\rm \bar \rho \bar l}}=1$ the continuity equation 
is Eq.~\ref{continuitysi}.
}
\\

{\color{orange} In the c.g.s.-Gaussian system the unit of current density  
(symbol ${\rm \bar J_{cgs}}$) is derived from its definition 
${\rm \bar J_{cgs}}={\rm statA}/{\rm cm}^2 = {1 \over \mathcal{K} 10^{-4}} {\rm A}/{\rm m}^2=\curtocur\ {\rm A}/{\rm m}^2$.
\\
Since in Eq.~\ref{continuitygen} ${{\rm \bar t_{cgs} \bar J_{cgs}} \over 
{\rm \bar \rho_{cgs} \bar l_{cgs}}}=1$ the continuity equation is 
Eq.~\ref{continuitysi}.
}
\\

%{\color{violet} In the modified c.g.s.-Gaussian system the unit of current 
%density  is derived from its definition and it is:
%$abA/cm^2 = {1 \over \mathcal{K}_A 10^{-4}} A/m^2=\curtocurm\ A/m^2$.
%\\
%The continuity equation can be derived by considering
%a charge density $\rho\ statC/cm^3= {1\over 10^{-6} \mathcal{K}}\ \rho\ C/m^3$ and
%a current density ${\bf J}\ abA/cm^2= {1 \over \mathcal{K}_A 10^{-4}} {\bf J}\ A/m^2.$
%Accounting also for the different length units and 
%using Eq.~\ref{continuitysi} we find:
%\begin{equation}
%{1\over 10^{-6} \mathcal{K}} {\partial \rho \over \partial t}=-{1\over 10^{-6} \mathcal{K}_A} \nabla \cdot {\bf J},
%\label{continuitycgsm0}
%\end{equation}
%or in modified c.g.s.-Gaussian units:
%\begin{equation}
%{1\over c} {\partial \rho \over \partial t}=-\nabla \cdot {\bf J}.
%\label{continuitycgsm1}
%\end{equation}
%\\
%}

{\color{green} 
The conversion factor between the a.u. and the c.g.s.-Gaussian unit
is: ${\rm \bar J} = \barcurcgs\ {\rm statA}/{\rm cm}^2 $
\\

%The conversion factor between a.u. and modified c.g.s.-Gaussian units
%is $\bar J = \barcurcgsm\ abA/cm^2. \label{cgsmcur}$
}

\newpage
{\color{coral}\section{Electric field}}
\color{black}
In the SI the unit of the electric field is derived from 
its definition:

\begin{tcolorbox}
\begin{equation}
{\bf E}={{\bf F} \over q},
\end{equation}
\end{tcolorbox}

so the unit of the electric field is ${\rm N}/{\rm C}= {{\rm kg}\cdot {\rm m} \over {\rm s}^2 \cdot {\rm C}}
={\rm V}/{\rm m}$.
\\

{\color{web-blue} In a.u. the unit of the electric field (symbol ${\rm \bar E}$) 
is derived from its definition ${\rm \bar E}={{\rm \bar f} \over {\rm \bar C}}$. The 
conversion factor to the SI unit is:
\begin{equation}
{\rm \bar E}= {E_h\over a_B e}= {\barf\ {\rm N} \over \barc\ {\rm C}}= \bare\ {\rm N}/{\rm C}. 
\end{equation}
}
\\

{\color{orange} In the c.g.s.-Gaussian system the unit of electric field
(symbol ${\rm statV}/{\rm cm}$) is derived from its definition:
${\rm \bar E_{cgs}}={\rm statV}/{\rm cm}={\rm dyne} / {\rm statC} = {10^{-5}\ \mathcal{K} }\ {\rm N}/{\rm C}=\etoe\ {\rm N}/{\rm C}$. 
\\
}

{\color{green} 
The conversion factor between the a.u. and the c.g.s.-Gaussian unit
is: ${\rm \bar E} = \barecgs \ {\rm statV}/{\rm cm}$.
}

\newpage
{\color{coral}\section{Electric potential}}
\color{black}
In the SI the unit of the electric potential (symbol ${\rm V}$)
is derived from its definition. The electric 
potential is a function $\phi({\bf r})$ such that:

\begin{tcolorbox}
\begin{equation}
{\bf E}=- \nabla \phi({\bf r}),
\end{equation}
\end{tcolorbox}

so the unit of the electric potential is ${\rm V}={{\rm N} \cdot {\rm m} \over {\rm C}}={{\rm J}\over {\rm C}}$. 
This unit is called volt.
\\

{\color{web-blue} In a.u. the unit of the electric potential 
(symbol ${\rm \bar V}$) 
is derived from its definition so ${\rm \bar V}={{\rm \bar E} \cdot {\rm \bar l}}$. 
The conversion factor to the SI unit is:
\begin{equation}
{\rm \bar V}= {E_h\over e}=
{\baru\ {\rm N}\cdot {\rm m}\over \barc\ {\rm C}} =\barphi\ {\rm V}.
\end{equation}
}
\\

{\color{orange} In the c.g.s.-Gaussian system the unit of electric potential
(symbol ${\rm statV}$) is derived from its definition: 
${\rm \bar V_{cgs}}={\rm statV}={\rm dyne}\cdot {\rm cm} / {\rm statC} = {10^{-7}\ \mathcal{K}}\ {\rm N}\cdot {\rm m}/{\rm C}=\phitophi\ {\rm V}$. This unit
is called statvolt. Note that 
${\rm statV}=\sqrt{\rm dyne}$.
\\
}

{\color{green} 
The conversion factor between the a.u. and the c.g.s.-Gaussian unit
is:
${\rm \bar V} = \barphicgs \ {\rm statV}$.
}
\\

{\color{red} A commonly used unit of energy is the electron volt (symbol ${\rm eV}$)
defined as the energy acquired by an electron accelerated through a
potential difference of $1 {\rm V}$. Therefore ${\rm eV}=\e\ {\rm C}\cdot {\rm V}$ (or ${\rm J}$).
The Hartree energy espressed in ${\rm eV}$ is: 
\begin{equation}
E_h=\barphi\ {\rm eV},
\end{equation}
while the Rydberg energy expressed in eV is:
\begin{equation}
{E_h \over 2}=\ryev\ {\rm eV}.
\end{equation}
These units are used also to measure the frequency giving the energy
of a photon of frequency $\nu$, that is $h \nu$ instead of $\nu$.
}

\newpage
{\color{coral}\section{Capacitance}}
\color{black}
In the SI the unit of the capacitance (symbol ${\rm F}$) is derived from its 
definition. The capacitance of a capacitor is the ratio between the charge
on its surfaces and the voltage applied between them:

\begin{tcolorbox}
\begin{equation}
C=q/V,
\end{equation}
\end{tcolorbox}

so the unit of the capacitance is ${\rm F}={{\rm C} \over {\rm V}}={{\rm A} \cdot {\rm s} \over {\rm V}}=
{{\rm C}^2 \over {\rm N} \cdot {\rm m}}
={{\rm C}^2 \over {\rm J}}$. This unit is called farad.
\\

{\color{web-blue} In a.u. the unit of the capacitance (symbol ${\rm \bar F}$) 
is derived from its definition: ${\rm \bar F}={{\rm \bar C} \over {\rm \bar V}}$. The 
conversion factor with the SI unit is:
\begin{equation}
{\rm \bar F} = {e^2 \over E_h} = {(\barc\ {\rm C})^2\over \baru\ {\rm J}}=\barcap\ {\rm F}.
\end{equation}
}
\\

{\color{orange} In the c.g.s.-Gaussian system the unit of capacitance
is derived from its definition: ${\rm \bar C_{cgs}}={\rm statC}/{\rm statV}=
{\rm cm}$. The 
conversion factor with the SI unit is:
${\rm cm}={1\over 10^{-7}\ \mathcal{K}^2}\ {\rm C}/{\rm V}=\captocap\ {\rm F}$.
}
\\

{\color{green} 
The conversion factor between the a.u. and the c.g.s.-Gaussian unit is: 
${\rm \bar F} = \barcapcgs \ {\rm cm}$.
}


\newpage
{\color{coral}\section{Vacuum electric permittivity}}
\color{black}
In the SI the unit of the vacuum electric permittivity $\epsilon_0$
can be derived from the Coulomb law:

\begin{tcolorbox}
\begin{equation}
{\bf F} = {1\over 4 \pi \epsilon_0} {qq'\over r^3} {\bf r},
\end{equation}
\end{tcolorbox}

so the unit of $\epsilon_0$ is ${{\rm C}^2 \over {\rm N}\cdot {\rm m}^2}={{\rm F}\over {\rm m}}$. 
Its numerical
value in these units is given in Eq.~\ref{epsilon0}.
\\

{\color{web-blue} In a.u. $\epsilon_0$ is not used.} 
\\

{\color{orange} In the c.g.s.-Gaussian system $\epsilon_0$ is not used.
}
\\

{\color{green} In a.u. $\epsilon_0$ is not used. 
\\
}

\newpage
{\color{coral}\section{Electric dipole moment}}
\color{black}
In the SI the unit of the electric dipole moment of a localized
charge density is derived from its definition:

\begin{tcolorbox}
\begin{equation}
{\bf \wp}=\int_V {\bf r} \rho({\bf r}) d^3r,
\end{equation}
\end{tcolorbox}

so the units of the electric dipole moment are ${\rm C}\cdot {\rm m}$.
\\

{\color{web-blue} In a.u. the unit of electric dipole moment is derived
from its definition ${\rm \bar \wp}={\rm \bar C}\cdot {\rm \bar l}$. The conversion factor to the
SI unit is:
\begin{equation}
{\rm \bar \wp} = e\ a_B = \e\ {\rm C}\ \abohr {\rm m} = \bardip\ {\rm C}\cdot {\rm m}.
\end{equation}
} 
\\

{\color{orange} In the c.g.s.-Gaussian system the unit of electric 
dipole is derived from its definition ${\rm \bar \wp_{cgs}}={\rm statC}\cdot {\rm cm}$.
The conversion factor to the SI unit is ${\rm \bar \wp_{cgs}}={10^{-2} \over 
\mathcal{K}}
\ {\rm C}\cdot {\rm m}=\diptodip\ {\rm C}\cdot {\rm m}$.
}
\\

{\color{green} 
The conversion factor between the a.u. and the c.g.s.-Gaussian unit is: 
${\rm \bar \wp} = \bardipcgs \ {\rm statC}\cdot {\rm cm}$.
}

\newpage
{\color{coral}\section{Polarization}}
\color{black}
In the SI the unit of the polarization is derived from its definition as
the electric dipole per unit volume ${{\rm C} \over {\rm m}^2}$: 

\begin{tcolorbox}
\begin{equation}
{\bf P}= {\bf \wp} / V.
\end{equation}
\end{tcolorbox}

{\color{web-blue} In a.u. the unit of the polarization (symbol ${\rm \bar P}$) 
is derived from its definition ${\rm \bar P}={{\rm \bar C} \over {\rm \bar l}^2}$. The 
conversion factor with the SI unit is:
\begin{equation}
{\rm \bar P} = {e \over a_B^2} =
{\barc\ {\rm C} \over (\barl\ {\rm m})^2} = \barpolar\ {\rm C}/{\rm m}^2.
\end{equation}
}
\\

{\color{orange} In the c.g.s.-Gaussian system the unit of polarization
is derived from its definition:
${\rm \bar P_{cgs}}={\rm statC}/{\rm cm}^2 = {1\over \mathcal{K}\ 10^{-4}}\ {\rm C}/{\rm m}^2=\polartopolar\ {\rm C}/{\rm m}^2$. 
\\
}

{\color{green} 
The conversion factor between the a.u. and the c.g.s.-Gaussian unit is: 
${\rm \bar P} = \barpolarcgs \ {\rm statC}/{\rm cm}^2$.
}

\newpage
{\color{coral}\section{Electric displacement}}
\color{black}
In the SI the electric displacement is given in term of the
electric field and of the polarization by:

\begin{tcolorbox}
\begin{equation}
{\bf D}=\epsilon_0 {\bf E} + {\bf P},
\label{eldisp}
\end{equation}
\end{tcolorbox}

so the polarization and the electric displacement have the same
unit ${{\rm C} \over {\rm m}^2}$. \\
This equation is usually justified starting from
the Maxwell's equation:

\begin{tcolorbox}
\begin{equation}
\nabla \cdot {\bf E} = {\rho \over \epsilon_0},
\end{equation}
\end{tcolorbox}

and separating $\rho$ into $\rho=\rho_f + \rho_b$ where
$\rho_f$ are the free charges and $\rho_b$ are the 
bound charges such that:

\begin{tcolorbox}
\begin{equation}
\rho_b=-\nabla \cdot {\bf P}.
\end{equation}
\end{tcolorbox}

Therefore one obtains:

\begin{tcolorbox}
\begin{equation}
\nabla \cdot (\epsilon_0 {\bf E}+ {\bf P})=\rho_f. 
\end{equation}
\end{tcolorbox}

{\color{web-blue} In a.u. the unit of the electric displacement 
(symbol ${\rm \bar D}$) is derived from the 
relationship between electric displacement, electric field, and polarization
that can be found writing the above equations in a.u.. The Maxwell's
equation becomes:
\begin{equation}
{{\rm \bar E} \over {\rm \bar l}} \nabla \cdot {\bf E} = {{\rm \bar \rho} \over \epsilon_0}
\rho
\label{maxwell1gen}
\end{equation}
and since ${{\rm \bar \rho}\cdot {\rm  \bar l} \over \epsilon_0 {\rm \bar E}}=4\pi$, in a.u. we have:
\begin{equation}
\nabla \cdot {\bf E} = 4 \pi \rho.
\end{equation}
The link between bound charges and polarization does not change:
\begin{equation}
\rho_b = -{{\rm \bar P} \over {\rm \bar l} \cdot {\rm \bar \rho}} \nabla \cdot {\bf P}=
-\nabla \cdot {\bf P},
\end{equation}
where we used the fact that ${{\rm \bar P} \over {\rm \bar l}\cdot {\rm \bar \rho}}=1$.
Inserting this expression in the Maxwell's equation we obtain:
\begin{equation}
\nabla \cdot ({\bf E}+ 4 \pi {\bf P})=4 \pi \rho_f, 
\end{equation}
that suggests the following definition of the electric displacement:
\begin{equation}
{\bf D}={\bf E} + 4\pi {\bf P}.
\label{eldispau}
\end{equation}

Finally we can find ${\rm \bar D}$ by using Eq.~\ref{eldisp} for an electric
field ${\bf E} {\rm \bar E}$, a polarization ${\bf P} {\rm \bar P}$ and a
displacement ${\bf D} {\rm \bar D}$:
\begin{equation}
{\bf D} = \left[{\bf E}\ {\epsilon_0 {\rm \bar E}\over {\rm \bar D}} + 
{\bf P} {{\rm \bar P}\over {\rm \bar D}}\right].
\label{eldispgen}
\end{equation}
Comparing this equation with Eq.~\ref{eldispau} we obtain:
\begin{equation}
{\rm \bar D}= {\epsilon_0 {\rm \bar E}}= {{\rm \bar P} \over 4 \pi}
\end{equation}
or ${\rm \bar D}= {{\rm \bar P}\over 4 \pi}=\bard\ {\rm C}/{\rm m}^2$.
\\
}

{\color{orange} In the c.g.s.-Gaussian system the relationship
between electric displacement, electric field, and polarization can 
be found noticing that in Eq.~\ref{maxwell1gen}
${{\rm \bar \rho_{cgs}}\cdot {\rm \bar l_{cgs}} \over \epsilon_0 {\rm \bar E_{cgs}}}=4\pi$
so that the Maxwell's equation becomes:
\begin{equation}
\nabla \cdot {\bf E} = 4 \pi \rho.
\end{equation}
Moreover since ${{\rm \bar P_{cgs}} \over {\rm \bar l_{cgs}}\cdot {\rm \bar \rho_{cgs}}}=1$,
the link between bound charge and polarization is as in the SI units and
we obtain:
\begin{equation}
\nabla \cdot ({\bf E}+ 4 \pi {\bf P})=4 \pi \rho_f, 
\end{equation}
that suggests the following relationship between electric displacement,
electric field and polarization:
\begin{equation}
{\bf D} =  {\bf E} + 4 \pi {\bf P}.
\label{eldispcgs}
\end{equation}
\\
The conversion factor with the SI unit ${\rm \bar D_{cgs}}$ can be found
from Eq.~\ref{eldispgen} that compared with Eq.~\ref{eldispcgs} gives:
\begin{equation}
{\rm \bar D_{cgs}}= {\epsilon_0 {\rm \bar E_{cgs}}}= {{\rm \bar P_{cgs}} \over 4 \pi}.
\end{equation}
The electric displacement and the polarization 
have the same dimensions ${\rm statC}/{\rm cm}^2$ but while the unit of polarization
is ${\rm \bar P_{cgs}}=1\ {\rm statC}/{\rm cm}^2$ the unit of electric displacement 
is ${\rm \bar D_{cgs}}={1\over 4 \pi}\ {\rm statC}/{\rm cm}^2$.
The conversion factor to SI units is:
\begin{equation}
{\rm \bar D_{cgs}}= {1\over 4 \pi\ \mathcal{K}\ 10^{-4}}\ {\rm C}/{\rm m}^2=\dtod\ {\rm C}/{\rm m}^2. 
\end{equation}
}
\\

{\color{green} 
The conversion factor between the a.u. and the c.g.s.-Gaussian unit is: 
${\rm \bar D} = \bardcgs \ {\rm statC}/({\rm cm}^2 \cdot 4 \pi)$.
%=\bardcgsoverfpi \ statC/cm^2$.
}

\newpage
{\color{coral}\section{Resistance}}
\color{black}
In the SI the unit of the resistance (symbol $\Omega$)
is derived from its definition. The resistance is the ratio between 
the applied voltage and the current that passes through a system:

\begin{tcolorbox}
\begin{equation}
R=V/I,
\end{equation}
\end{tcolorbox}

so the unit of the resistance is $\Omega={{\rm V} \over {\rm A}}=
{{\rm N}\cdot {\rm m} \cdot {\rm s} \over {\rm C}^2}=
{{\rm J}\cdot {\rm s} \over {\rm C}^2}$. This
unit is called ohm.
\\

{\color{web-blue} In a.u. the unit of the resistance (symbol $\bar \Omega$) 
is derived from its definition: $\bar \Omega={{\rm \bar V} \over {\rm \bar I}}$. The
conversion factor with the SI unit is:
\begin{equation}
\bar \Omega = {E_h \hbar \over e^2 E_h}=  
{\hbar \over e^2}=
{\hbarf\ {\rm J}\cdot {\rm s}\over (\barc\ {\rm C})^2}=\barohm\ \Omega.
\end{equation}
\\
}

{\color{orange} In the c.g.s.-Gaussian system the unit of resistance
is derived from its definition:
${\rm \bar \Omega_{cgs}}={\rm statV}/{\rm statA}={\rm s}/{\rm cm}$. The conversion factor with the SI 
unit is:
${\rm \bar \Omega_{cgs}}=10^{-7}\ \mathcal{K}^2\ {\rm V}/{\rm A}=\ohmtoohm\ {\Omega}$.
}
\\

{\color{green}
The conversion factor between the a.u. and the c.g.s.-Gaussian unit
is: $\bar \Omega = \barohmcgs \ {\rm s}/{\rm cm}$.
}


\newpage
{\color{coral}\section{Magnetic flux density}}
\color{black}
In the SI the unit of the magnetic flux density 
(symbol ${\rm T}$) can be derived from the Lorentz force equation:

\begin{tcolorbox}
\begin{equation}
{\bf F}=q ({\bf v} \times {\bf B}),
\label{lforce}
\end{equation}
\end{tcolorbox}

so the unit of the magnetic flux density is ${\rm T}={{\rm N}\cdot {\rm s}\over {\rm C}\cdot {\rm m}} = 
{{\rm V} \cdot {\rm s} \over {\rm m}^2}={{\rm N} \over {\rm A}\cdot {\rm m}}={{\rm kg} \over {\rm s} \cdot {\rm C}}$. 
This unit is called tesla. \\
In the SI the second Maxwell's equation reads:

\begin{tcolorbox}
\begin{equation}
\nabla \times {\bf E}=-{\partial {\bf B} \over \partial t}.
\end{equation} 
\end{tcolorbox}

{\color{web-blue} In a.u. the unit of the magnetic flux density 
(symbol ${\rm \bar B}$) is derived from the Lorentz force: 
${\rm \bar B}= {{\rm \bar f} \over {\rm \bar v} \cdot
{\rm \bar C}}$. The conversion factor with the SI unit is:
\begin{align}
{\rm \bar B} = {E_h \hbar \over a_B^2 E_h e}={\hbar \over a_B^2 e}&=
{\hbarf\ {\rm J}\cdot {\rm s} \over (\barl\ {\rm m})^2\ \barc\ {\rm C}}\nonumber
\\&= \barb\ {\rm T}.
\end{align}
\\

Using the expression of $a_B$ we can also write:
\begin{equation}
{\rm \bar B}= {m_e e \over a_B \hbar 4 \pi \epsilon_0}=
{E_h \over 2 \mu_B},
\end{equation}
where $E_h$ is the Hartree energy (Eq.\ref{eh}) and $\mu_B$ is the Bohr 
magneton (Eq.~\ref{bohrm}).
\\
In these units the second Maxwell's equation is:
\begin{equation}
\nabla \times {\bf E}=-{{\rm \bar l}\cdot {\rm \bar B} \over {\rm \bar t}\cdot {\rm \bar E}} 
{\partial {\bf B} \over \partial t}.
\label{maxwell2gen}
\end{equation}
Since ${{\rm \bar l}\cdot {\rm \bar B} \over {\rm \bar t} \cdot {\rm \bar E}}=1$, the second Maxwell's
equation has the same form as in the SI.
}
\\

{\color{orange} In the c.g.s.-Gaussian system the unit of magnetic
flux density (symbol ${\rm G}$) is derived from the Amp\`ere law by writing 
the force per unit length between two parallel wires traversed by 
a current $I$ as:
\begin{equation}
{F \over l} = {1\over c} B \cdot I,
\end{equation}
where $B={2\over c} {I \over r}$ is the magnetic flux density produced by 
one wire at the distance of the other. 
This relationship shows that ${\bf B}$ has the dimensions of a
charge per unit area ${\rm statC}/{\rm cm}^2$. This unit is called gauss.

The conversion factor with the SI unit can be found using the expression of 
the magnetic flux density $B$ produced by a wire crossed by 
a currenit $I$ in SI units: $B={\mu_0\over 2 \pi} {I \over r}$. 
Writing ${\rm G}={1\over \mathcal{K}_T} {\rm T}$, a current of $I\ {\rm statA}= {I\over \mathcal{K}} 
{\rm A}$ produces at a distance of $r\ {10^{-2}} {\rm m}$ a magnetic field 
(in ${\rm T}=\mathcal{K}_T {\rm G}$): 
\begin{equation}
B={\tilde \mu_0\over 2 \pi}\ {\mathcal{K}_T \over
10^{-2} \mathcal{K}}\ {I\over r}\  {\rm G}.
\end{equation}
Comparing with the c.g.s.-Gaussian expression we obtain:
\begin{equation}
\mathcal{K}_T= {4 \pi \over \tilde \mu_0} {10^{-2}\ \mathcal{K} \over \tilde c_{cgs}}= 
{10^3 \over \mathcal{K}_A}=\toverg,
\end{equation}
where $\mathcal{K}_A=\mathcal{K}/\tilde c_{cgs}=\kappaa$. 
With the old SI units $\mathcal{K}_T$ was exactly $10^4$.
\\
Using the conversion factor between ${\rm T}$ and ${\rm G}$ we can write the 
Lorentz force in the c.g.s-Gaussian system. A particle with charge
$q\ {\rm statC}={q\over \mathcal{K}}\ {\rm C}$ that moves with a speed of ${\bf v}\ {\rm cm}/{\rm s}=
{\bf v}\ 10^{-2}\ {\rm m}/{\rm s}$ in a field of ${\bf B}\ {\rm G} = {{\bf B} \over \mathcal{K}_T}\ {\rm T}$ 
will fill a force (in ${\rm N}=10^5\ {\rm dyne}$) (using Eq.~\ref{lforce}):
\begin{equation}
{\bf F}={10^{-2} \over \mathcal{K}\ \mathcal{K}_T} q\ ({\bf v}\times {\bf B})\ 10^5\ {\rm dyne}. 
\end{equation}
Since ${10^{3} \over \mathcal{K} \mathcal{K}_T}=
{1\over \tilde c_{cgs}}$ we obtain the Lorentz force in the 
c.g.s.-Gaussian system:
\begin{equation}
{\bf F}={q\over c}\ ({\bf v}\times {\bf B}).
\end{equation}
The second Maxwell's equation can be found noticing that 
in Eq.~\ref{maxwell2gen}
${{\rm \bar l_{cgs}}\cdot {\rm \bar B_{cgs}} \over {\rm \bar t_{cgs}}\cdot {\rm \bar E_{cgs}}}=
{10^{-2} \over \mathcal{K}_T 10^{-5} \mathcal{K}}={10^3 \mathcal{K}_A \over 10^3 \mathcal{K}}={1\over 
\tilde c_{cgs}}$ so we have:
\begin{equation}
\nabla \times {\bf E}=-{1\over c}
{\partial {\bf B} \over \partial t}.
\label{maxwell2cgs}
\end{equation}
}
\\

%{\color{violet} In the modified c.g.s.-Gaussian system the unit of 
%magnetic flux density is derived from the Amp\`ere law by writing the force per
%unit length between two parallel wires as:
%\begin{equation}
%{F \over l} = B \cdot I
%\end{equation}
%where $B=2 {I \over r}$ is the magnetic flux density. 
%In this units ${\bf B}$ has dimension $abA/cm=\sqrt{dyne}/cm=statC/cm^2$,
%so also in this system of units ${\bf B}$ is measured in $G$.
%With the same reasoning as before and using $A=\mathcal{K}_A\ abA$ we find:
%\begin{equation}
%{\tilde \mu_0\over 2 \pi} \mathcal{K}_T G= {2 \over 10^2} \mathcal{K}_A statC/cm^2
%\end{equation}
%that gives again
%\begin{equation}
% \mathcal{K}_T = {10^3 \over \mathcal{K}_A}=\toverg.
%\end{equation}
%\\
%}

{\color{green} 
The conversion factor between the a.u. and the c.g.s.-Gaussian unit is: 
${\rm \bar B} = \barbcgs\ {\rm G}$.
}

\newpage
{\color{coral}\section{Vector potential}}
\color{black}

In the SI the unit of the vector potential (symbol ${\rm T} \cdot {\rm m}$)
is derived from its definition. The vector potential (${\bf A}$)
is a vector field such that:

\begin{tcolorbox}
\begin{equation}
{\bf B}=\nabla \times {\bf A},
\label{defav}
\end{equation}
\end{tcolorbox}

so the unit of the vector potential is ${\rm T}\cdot {\rm m}={{\rm N} \over {\rm A}}=
{{\rm V}\cdot {\rm s} \over {\rm m}}={{\rm N}\cdot {\rm s} \over {\rm C}}$. \\
When the vector potential depends on time the electric field is:

\begin{tcolorbox}
\begin{equation}
{\bf E}=-\nabla \phi -{\partial {\bf A} \over \partial t}.
\label{ephia}
\end{equation}
\end{tcolorbox}

{\color{web-blue} In a.u. the unit of the vector potential
(symbol ${\rm \bar A}$) is derived from its definition ${\rm \bar A} = {\rm \bar B}\cdot {\rm \bar l}$.
We have:
\begin{align}
{\rm \bar A} = {\hbar \over a_B e}&= {\hbarf\ {\rm J}\cdot {\rm s} \over \abohr\ {\rm m}\ \e\ {\rm C}}
\nonumber \\ 
&= \barav\ {\rm T}\cdot {\rm m}.
\end{align}
When the vector potential depends on time the electric field is:
\begin{equation}
{\bf E}=-{{\rm \bar V} \over {\rm \bar E}\cdot {\rm \bar l}} \nabla \phi -
{{\rm \bar A} \over {\rm \bar E}\cdot {\rm \bar t}} {\partial {\bf A} \over \partial t}.
\label{ephiagen}
\end{equation}
Since ${{\rm \bar V} \over {\rm \bar E}\cdot {\rm \bar l}}=1$ and 
${{\rm \bar A} \over {\rm \bar E}\cdot {\rm \bar t}}=1$ Eq.~\ref{ephia} holds also in a.u..
}
\\

{\color{orange} In the c.g.s.-Gaussian system the unit of the vector 
potential (symbol ${\rm \bar A_{cgs}}$) is derived from its definition 
Eq.~\ref{defav}.
The conversion factor with the SI unit is 
${\rm \bar A_{cgs}} = {\rm G} \cdot {\rm cm} = {10^{-2}\over \mathcal{K}_T}\ {\rm T}\cdot {\rm m}=
{\mathcal{K}_A \over 10^5}\ {\rm T}\cdot {\rm m}=\avtoav\ {\rm T}\cdot {\rm m}$. \\

When the vector potential depends on time we can calculate the electric
field as in Eq.~\ref{ephiagen} and since 
${{\rm \bar V_{cgs}} \over {\rm \bar E_{cgs}}\cdot {\rm \bar l_{cgs}}}=1$
and ${{\rm \bar A_{cgs}} \over {\rm \bar E_{cgs}} \cdot {\rm \bar t_{cgs}}}={1\over 
\tilde c_{cgs}}$, in the 
c.g.s.-Gaussian system the electric field is given by:
\begin{equation}
{\bf E}=-\nabla \phi -
{1 \over c} {\partial {\bf A} \over \partial t}.
\end{equation}
}
\\

{\color{green}
The conversion factor between the a.u. and the c.g.s.-Gaussian unit is:
${\rm \bar A} = \baravcgs\ {\rm G}\cdot {\rm cm}$.
}

\newpage
{\color{coral}\section{Magnetic field flux}}
\color{black}

In the SI the unit of the magnetic field flux (symbol ${\rm Wb}$)
is derived from its definition. The  
magnetic field flux through a surface perpendicular to 
$\hat {\bf n}$ is:

\begin{tcolorbox}
\begin{equation}
\Phi=\int {\bf B}\cdot \hat {\bf n}\ dS,
\end{equation}
\end{tcolorbox}

so the unit of the magnetic field flux is ${\rm Wb}= {\rm T} \cdot {\rm m}^2={{\rm N} \cdot {\rm s}\cdot {\rm m}
\over {\rm C}}={\rm V}\cdot {\rm s}={{\rm N}\cdot {\rm m}\over {\rm A}}={{\rm kg}\cdot {\rm m}^2 \over {\rm s} {\rm C}}=
{{\rm J}\cdot {\rm s}\over {\rm C}}={{\rm J}\over {\rm A}}$. This
unit is called weber.
\\

{\color{web-blue} In a.u. the unit of the magnetic field flux
(symbol ${\rm \bar {Wb}}$) can be derived from its definition
${\rm \bar {Wb}}={\rm \bar B} \cdot {\rm \bar l}^2$. The conversion
factor with the SI unit is:
\begin{equation}
{\rm \bar {Wb}} = {\hbar \over e}=
{\hbarf\ {\rm J}\cdot {\rm s}\over \e\ {\rm C}}=\barwb\ {\rm Wb}.
\end{equation}
\\
}

{\color{orange} In the c.g.s.-Gaussian system the unit of magnetic field
flux (symbol ${\rm Mx}$) is derived from its definition: ${\rm Mx}={\rm G}\cdot {\rm cm}^2$.
This unit is called maxwell.
The conversion factor with the SI unit is:
\begin{equation}
{\rm Mx} = {10^{-4}\over \mathcal{K}_T}\ {\rm T}\cdot {\rm m}^2 = 10^{-7}\ \mathcal{K}_A\ {\rm Wb} =\wbtowb\ {\rm Wb}
\end{equation}
In the old SI units this factor was $10^{-8}$.
}
\\

{\color{green} 
The conversion factor between the a.u. and the c.g.s.-Gaussian unit is: 
$\bar Wb = \barwbcgs\ {\rm Mx}$.
}

\newpage
{\color{coral}\section{Inductance}}
\color{black}

In the SI the unit of the inductance (symbol ${\rm H}$)
is derived from its definition as the ratio of the magnetic
field flux and the current in a circuit:

\begin{tcolorbox}
\begin{equation}
L={\Phi \over I},
\end{equation}
\end{tcolorbox}

so the unit of inductance is ${\rm H}={{\rm Wb} \over {\rm A}}={{\rm T} \cdot {\rm m}^2 \over {\rm A}}=
{{\rm V}\cdot {\rm s} \over {\rm A}} = \Omega \cdot {\rm s}= {{\rm N} \cdot {\rm m} \over {\rm A}^2}={{\rm J}\over {\rm A}^2}=
{{\rm kg}\cdot {\rm m}^2 \over {\rm C}^2}$. This unit is called henry. 
\\

{\color{web-blue} In a.u. the unit of the inductance (symbol ${\rm \bar Y}$) 
can be derived from its definition as:
${\rm \bar Y}={{\rm \bar {Wb}} \over {\rm \bar I}}$. The conversion factor with the SI unit is:
\begin{align}
{\rm \bar Y} = {\hbar {\rm \bar t}\over e^2}={m_e a_B^2 \over e^2}={\hbar^2\over E_h e^2}
&= {(\hbarf\ {\rm J}\cdot {\rm s})^2 \over \baru\ {\rm J}\ (\e\ {\rm C})^2}\nonumber \\
&= \bary\ {\rm H}.
\end{align}
\\
}

{\color{orange} In the c.g.s.-Gaussian system the unit of inductance
(symbol ${\rm statH}$) is derived from its definition: 
\begin{equation}
L={1\over c} {\Phi \over I}.
\label{inductancecgs}
\end{equation}
This relationship shows that inductance has the dimensions of 
${\rm s}^2/{\rm cm}$. This unit is called stathenry.
The conversion factor with the SI unit can be found by writing
${\rm \bar Y_{cgs}}={\rm statH}$ and 
\begin{equation}
L = {{\rm \bar \Phi_{cgs}} \over {\rm \bar Y_{cgs}}\cdot {\rm \bar I_{cgs}}} {\Phi \over I}.
\end{equation}
Comparing with Eq.~\ref{inductancecgs} we obtain:
\begin{equation}
{\rm \bar Y_{cgs}}= {\tilde c_{cgs} {\rm \bar \Phi_{cgs}} \over {\rm \bar I_{cgs}}},
\end{equation}
therefore:
\begin{equation}
{\rm statH}=\tilde c_{cgs} {10^{-4} \mathcal{K} \over \mathcal{K}_T}\ {\rm H} =
{\tilde c_{cgs}} {10^{-7} \mathcal{K}_A \mathcal{K}}\ {\rm H}=
10^{-7} \mathcal{K}^2\ {\rm H} = \ytoy\ {\rm H}.
\end{equation}
\\
}

{\color{green} 
The conversion factor between the a.u. and the c.g.s.-Gaussian unit is: 
${\rm \bar H} = \barycgs\ {\rm statH}$.
}

\newpage
{\color{coral}\section{Magnetic dipole moment}}
\color{black}

In the SI the unit of the magnetic dipole moment
can be derived from its definition. For instance the magnetic dipole 
moment ${\boldsymbol \mu}$ of a coil traversed by a current $I$ is:

\begin{tcolorbox}
\begin{equation}
{\boldsymbol \mu}=IA \hat {\bf n},
\label{magdip}
\end{equation}
\end{tcolorbox}

where $A$ is the area of the coil and $\hat {\bf n}$ is a versor 
normal to the area. Therefore the unit of the magnetic dipole moment is 
${\rm A}\cdot {\rm m}^2={{\rm N} \cdot {\rm m} \cdot  {\rm A} \cdot {\rm m} \over {\rm N}}={\rm J}/{\rm T}$.  
\\

{\color{web-blue} In a.u. the unit of the magnetic dipole moment 
(symbol $\bar \mu$) is derived from its definition as 
$\bar \mu= {\rm \bar I} \cdot {\rm \bar l}^2$. The conversion factor with the SI unit
is: 
\begin{align}
\bar \mu = {{\rm \bar C} \cdot {\rm \bar l}^2 \over {\rm \bar t}}= 
{e a_B^2 \hbar \over m_e a_B^2}
= {\hbar e \over m_e}=2 \mu_B &=
{\hbarf\ {\rm J}\cdot {\rm s}\ \e\ {\rm C} \over \me\ {\rm kg}}\nonumber \\ 
&= \barmu {\rm J}/{\rm T}.
\end{align}
\\
}

{\color{orange} In the c.g.s.-Gaussian system the unit of magnetic dipole
moment (symbol ${\rm \bar \mu_{cgs}}$) is derived from its definition:
\begin{equation}
{\boldsymbol \mu}={1\over c} I A \hat {\bf n},
\label{magdipcgs}
\end{equation}
hence its dimensions are 
${\rm statC}\cdot {\rm cm}$. To find the conversion factor with the SI unit we use
Eq.~\ref{magdip} for a current $I\ {\rm \bar I_{cgs}}$ and area $A\ {\rm \bar l_{cgs}}^2$
that gives a dipole moment ${\rm \bar \mu_{cgs}} {\boldsymbol \mu}$. We have
therefore:
\begin{equation}
{\boldsymbol \mu} = {{\rm \bar I_{cgs}}\cdot {\rm \bar l_{cgs}}^2 \over \bar \mu_{cgs} }
I A \hat {\bf n}.
\end{equation} 
Comparing with Eq.~\ref{magdipcgs} we obtain:
\begin{equation}
\bar \mu_{cgs}= \tilde c_{cgs} {\rm \bar I_{cgs}}\cdot {\rm \bar l_{cgs}}^2= 
{\tilde c_{cgs} 10^{-4} \over \mathcal{K}}\ {\rm A}\cdot {\rm m}^2 = 
{1\over 10^4 \mathcal{K}_A}\ {\rm A}\cdot {\rm m}^2 = \mutomu\ {\rm A}\cdot {\rm m}^2.
\end{equation}
}
\\

%{\color{violet} In the modified c.g.s.-Gaussian system the unit of magnetic 
%dipole moment must be modified as:
%\begin{equation}
%{\boldsymbol \mu}={1 \over c}IA \hat {\bf n}
%\end{equation}
%and the unit derived from the definition is 
%$\bar \mu_{cgs}={1\over c} abA\cdot cm^2$.
%Using the conversion factors found above we have:
%\begin{equation}
%\bar \mu_{cgs} = {10^{-4}\over \tilde c_{cgs} \mathcal{K}_A}\ A\cdot m^2 
%={10^{-4}\over \mathcal{K}}\ A\cdot m^2 
%= \mutomu\ A\cdot m^2
%\end{equation}
%so in this modified system the units of the magnetic dipole moment 
%does not change.
%}
%\\

{\color{green} 
The conversion factor between the a.u. and the c.g.s.-Gaussian unit is: 
$\bar \mu = \barmucgs\ {\rm statC}\cdot {\rm cm}$.
}

\newpage
{\color{coral}\section{Magnetization}}
\color{black}

In the SI the unit of the magnetization can be derived from its definition
as the magnetic dipole moment per unit volume:

\begin{tcolorbox}
\begin{equation}
{\bf M} = {{\boldsymbol \mu}\over V},
\end{equation}
\end{tcolorbox}

where $V$ is the volume of the sample.
Therefore the unit of the magnetization is ${\rm A}\over {\rm m}$.  

{\color{web-blue} In a.u. the unit of the magnetization 
(symbol ${\rm \bar M}$) is derived from its definition
as ${\rm \bar M}= {{\rm \bar I} \over {\rm \bar l}}$. The conversion factor with
the SI unit is:
\begin{equation}
{\rm \bar M} = {\bari\ {\rm A} \over \barl\ {\rm m}} = \barmag {\rm A/m}.
\end{equation}
\\
Using the expression of $\bar \mu$ and ${\rm \bar l}$ we can write also:
\begin{equation}
\bar M = {2 \mu_B \over a_B^3}.
\end{equation}
}

{\color{orange} In the c.g.s.-Gaussian system the unit of magnetization
(symbol ${\rm \bar M_{cgs}}$) is derived from its definition: 
${\rm \bar M_{cgs}}={{\rm statC} \over {\rm cm}^2}$.
The conversion with the SI unit is:
\begin{equation}
{\rm \bar M_{cgs}} = {\bar \mu_{cgs} \over {\rm cm}^3}={1\over 10^{-2}\ \mathcal{K}_A}\ {\rm A}/{\rm m} =
\magtomag\ {\rm A}/{\rm m}.
\end{equation}
}
\\

{\color{green} The conversion factor between the a.u. and the 
c.g.s.-Gaussian unit is:
${\rm \bar M} = \barmagcgs\ {\rm statC}/{\rm cm}^2$.
}


\newpage
{\color{coral}\section{Vacuum magnetic permeability}}
\color{black}

In the SI the unit of the vacuum magnetic permeability $\mu_0$
be derived from the equation: 

\begin{tcolorbox}
\begin{equation}
\mu_0={1\over c^2 \epsilon_0},
\end{equation}
\end{tcolorbox}

where $c$ is the speed of light. Therefore the unit of $\mu_0$ is 
${{\rm N}\cdot {\rm s}^2 \over {\rm C}^2}={{\rm N} \over {\rm A}^2}={{\rm V}\cdot {\rm s} \over {\rm A} \cdot {\rm m}}=
{{\rm H}\over {\rm m}}={{\rm m}\cdot {\rm T} \over {\rm A}}$. Its numerical
value is given in Eq.~\ref{mu0} and it is approximately $4\pi\times 10^{-7}$.
\\

{\color{web-blue} In a.u. the vacuum magnetic permeability $\mu_0$ is not 
used.}
\\

{\color{orange} In the c.g.s.-Gaussian system $\mu_0$ is not used.
}
\\

%{\color{violet} In the generalized c.g.s.-Gaussian system $\mu_0$ is not used.
%}
%\\

\newpage
{\color{coral}\section{Magnetic field strength}}
\color{black}

In the SI the magnetic field strength is given in term of the
magnetic flux density and of the magnetization by:

\begin{tcolorbox}
\begin{equation}
{\bf H}={{\bf B} \over \mu_0} - {\bf M},
\label{magintensity}
\end{equation}
\end{tcolorbox}

so the magnetic field and the magnetization have the same
unit ${{\rm A} \over {\rm m}}$.\\
Eq.~\ref{magintensity} can be derived from the fourth Maxwell's equation:

\begin{tcolorbox}
\begin{equation}
\nabla \times {\bf B}= \mu_0 {\bf J} + {1\over c^2} {\partial {\bf E} \over
\partial t}.
\label{maxwell4}
\end{equation}
\end{tcolorbox}

Separating ${\bf J}={\bf J}_f + {\bf J}_m + {\bf J}_d$ where
${\bf J}_f$ is the current of the free charges, 

\begin{tcolorbox}
\begin{equation}
{\bf J}_m=\nabla \times {\bf M}
\end{equation}
\end{tcolorbox}

is the magnetization current, and

\begin{tcolorbox}
\begin{equation}
{\bf J}_d={\partial {\bf P} \over \partial t}
\end{equation}
\end{tcolorbox}

is the displacement current and inserting these expressions in 
Eq.~\ref{maxwell4} one obtains, after dividing by $\mu_0$:

\begin{tcolorbox}
\begin{equation}
\nabla \times ({{\bf B}\over \mu_0}-{\bf M})= {\bf J}_f + 
{\partial ({\bf P} + \epsilon_0 {\bf E} )\over
\partial t}.
\end{equation}
\end{tcolorbox}

Using Eq.~\ref{magintensity} and the definition of ${\bf D}$ the
fourth macroscopic Maxwell's equation becomes:

\begin{tcolorbox}
\begin{equation}
\nabla \times {\bf H}= {\bf J}_f + 
{\partial {\bf D}\over
\partial t}.
\end{equation}
\end{tcolorbox}

{\color{web-blue} In a.u. the unit of magnetic field strength (${\rm \bar H}$) 
can be derived from the relationship between magnetic field strength,
magnetic flux density, and magnetization. This relationship can
be derived writing Eq.~\ref{maxwell4} in a.u..
We have:
\begin{equation}
\nabla \times {\bf B}= {\mu_0 {\rm \bar J}\cdot {\rm \bar l} \over {\rm \bar B}} {\bf J} + 
{{\rm \bar E}\cdot {\rm \bar l} \over c^2 {\rm \bar t}\cdot {\rm \bar B}} {\partial {\bf E} \over
\partial t}.
\label{maxwell4gen}
\end{equation}
Since ${\mu_0 {\rm \bar J}\cdot {\rm \bar l} \over {\rm \bar B}}=4 \pi \alpha^2$ and 
${{\rm \bar E}\cdot {\rm \bar l} \over c^2 {\rm \bar t}\cdot {\rm \bar B}}=\alpha^2$, where $\alpha$ 
is the fine structure constant, this equation becomes:
\begin{equation}
\nabla \times {\bf B}= {4 \pi \alpha^2} {\bf J} + {\alpha^2} 
{\partial {\bf E} \over \partial t}.
\label{maxwell4au}
\end{equation}
Moreover we have:
\begin{equation}
{\bf J}_m={{\rm \bar M} \over {\rm \bar l}\cdot {\rm \bar J}} \nabla \times {\bf M}
\end{equation}
and 
\begin{equation}
{\bf J}_d={{\rm \bar P} \over {\rm \bar t} \cdot {\rm \bar J}} {\partial {\bf P} \over \partial t}.
\end{equation}
Since ${{\rm \bar M} \over {\rm \bar l}\cdot {\rm \bar J}}=1$ and 
${{\rm \bar P} \over {\rm \bar t}\cdot {\rm \bar J}}=1$ we have:
\begin{equation}
{\bf J}_m= \nabla \times {\bf M}
\end{equation}
and 
\begin{equation}
{\bf J}_d={\partial {\bf P} \over \partial t},
\end{equation}
as in the SI. Therefore after division by $\alpha^2$ 
Eq.~\ref{maxwell4au} becomes:
\begin{equation}
\nabla \times ({{\bf B}\over \alpha^2} -4\pi {\bf M})= 
{4 \pi} {\bf J}_f +  {\partial ({\bf E}+4\pi {\bf P}) \over \partial t}.
\label{maxwell4au1}
\end{equation}
This equation suggests the definition of the magnetic field strength as:
\begin{equation}
{\bf H}={{\bf B}\over \alpha^2} -4\pi {\bf M},
\label{HBMau}
\end{equation}
that gives the macroscopic Maxwell's equation:
\begin{equation}
\nabla \times {\bf H}= 
{4 \pi} {\bf J}_f +  {\partial {\bf D} \over \partial t}.
\end{equation}
Finally, we can find ${\rm \bar H}$ by writing:
\begin{equation}
{\bf H} = {{\rm \bar B} \over \mu_0 {\rm \bar H}} {\bf B} - {{\rm \bar M} \over {\rm \bar H}} {\bf M}.
\label{HBMgen}
\end{equation}
Comparing with Eq.~\ref{HBMau} we obtain:
\begin{equation}
{\rm \bar H}={{\rm \bar B} \alpha^2 \over \mu_0} = {{\rm \bar M} \over 4 \pi}.
\end{equation}
Therefore the conversion factor with the SI unit is:
\begin{equation}
\bar H = \barh\ {\rm A}/{\rm m}.
\end{equation}
}
\\

{\color{orange} In the c.g.s.-Gaussian system the unit of the magnetic field
strength (symbol ${\rm Oe}$) is derived from the relationship between 
magnetic field strength,
magnetic flux density, and magnetization. This relationship can be derived
as discussed for the a.u. case. In Eq.~\ref{maxwell4gen} we have
${\mu_0 {\rm \bar J_{cgs}}\cdot {\rm \bar l_{cgs}} \over {\rm \bar B_{cgs}}}={4 \pi \over c_{cgs}}$
and ${{\rm \bar E_{cgs}}\cdot {\rm \bar l_{cgs}} \over c^2 {\rm \bar t_{cgs}}
\cdot {\rm \bar B_{cgs}}}=
{1\over c_{cgs}}$ so the Maxwell's equation becomes:
\begin{equation}
\nabla \times {\bf B}= {4 \pi \over c} {\bf J} + {1\over c} 
{\partial {\bf E} \over \partial t}.
\label{maxwell4cgs}
\end{equation}
Moreover we have:
${{\rm \bar M_{cgs}} \over {\rm \bar l_{cgs}}\cdot {\rm \bar J_{cgs}}}={\mathcal{K}\over \mathcal{K}_A}=c_{cgs}$ and
${{\rm \bar P_{cgs}} \over {\rm \bar t_{cgs}}\cdot {\rm \bar J_{cgs}}}=1$ so in c.g.s. units:
\begin{equation}
{\bf J}_m= c \nabla \times {\bf M}
\end{equation}
and
\begin{equation}
{\bf J}_d={\partial {\bf P} \over \partial t}.
\end{equation}
Inserting these expressions in the Maxwell's equation we have:
\begin{equation}
\nabla \times ({\bf B} -4\pi {\bf M})= 
{4 \pi\over c} {\bf J}_f +  {1\over c} {\partial ({\bf E}+4\pi {\bf P}) \over \partial t}.
\label{maxwell4cgs1}
\end{equation}
This equation suggests the definition of the magnetic field strength as:
\begin{equation}
{\bf H}={\bf B} -4\pi {\bf M},
\label{magintensitycgs}
\end{equation}
that gives the macroscopic Maxwell's equation:
\begin{equation}
\nabla \times {\bf H}= 
{4 \pi \over c} {\bf J}_f +  {1\over c}{\partial {\bf D} \over \partial t}.
\end{equation}

Comparing Eq.~\ref{magintensitycgs} and Eq.~\ref{HBMgen}, we can find the 
conversion factor with the SI unit.
We have ${\rm Oe}={{\rm \bar B_{cgs}} \over \mu_0}={{\rm \bar M_{cgs}} \over 4 \pi}$ or
\begin{equation}
{\rm Oe}= {1\over 4 \pi\ 10^{-2} \mathcal{K}_A} {\rm A}/{\rm m}=\htoh\ {\rm A}/{\rm m}. 
\end{equation}
This unit is called oersted.
}
\\

%{\color{violet} In the generalized c.g.s.-Gaussian system the unit 
%of magnetization
%and of magnetic field intensity do not coincide. The two quantities
%have the same dimensions $abA/cm$ but while the unit of magnetization is
%of $\bar M_{cgs}=1\ abA/cm$ the unit of the magnetic field intensity 
%is $\bar H_{cgs}={1\over 4 \pi} abA/cm$ 
%and in SI units it is therefore
%\begin{equation}
%\bar H_{cgs}= {1\over 4 \pi\ \mathcal{K}_A\ 10^{-2}} A/m=\htoh\ A/m
%\end{equation}
%\\
%Due to this definition we can write Eq.~\ref{magintensity} as:
%\begin{equation}
%{{\bf H}\over \bar H_{cgs}}\ \bar H_{cgs} = {\bar \mu_{0,cgs} \over 
% \mu_0} {{\bf B}\over \bar B_{cgs}}\
%{\bar B_{cgs} \over \bar \mu_{0,cgs} }- {{\bf M}\over \bar M_{cgs}} 
%\bar M_{cgs}
%\end{equation}
%and since $\bar H_{cgs} ={1\over 4\pi} \bar M_{cgs}$, ${\bar B_{cgs}
%\over \bar \mu_{0,cgs}}= \bar M_{cgs}$ and ${\bar \mu_{0,cgs} \over 
%\bar \mu_{0}}={1\over 4 \pi}$ in c.g.s.-Gaussian units we have
%\begin{equation}
%{\bf H} =  {\bf B} - 4 \pi {\bf M}
%\label{ldispcgs}
%\end{equation}
%}
%\\


{\color{green} The conversion factor between the a.u. and the c.g.s.-Gaussian 
unit is:
${\rm \bar H} = \barhcgs\ {\rm statC}/{\rm cm}^2$.
}

\newpage
{\color{coral}\section{Microscopic Maxwell's equations}}
\color{black}

In the SI the microscopic Maxwell's equations are:

\begin{tcolorbox}
\begin{align}
\nabla \cdot {\bf E} &= {\rho\over {\epsilon_0}}, \\
\nabla \times {\bf E} &= -{\partial {\bf B} \over \partial t}, \\
\nabla \cdot {\bf B} &= 0, \\
\nabla \times {\bf B} &= \mu_0{\bf J}+ {1\over c^2}
{\partial {\bf E} \over \partial t}. 
\end{align}
\end{tcolorbox}

{\color{web-blue} The form of the microscopic Maxwell's equations
in a.u. has been obtained in the previous text:
\begin{align}
\nabla \cdot {\bf E} &= 4 \pi \rho, \\
\nabla \times {\bf E} &= -{\partial {\bf B} \over \partial t}, \\
\nabla \cdot {\bf B} &= 0, \\
\nabla \times {\bf B} &= {4 \pi \over c^2} {\bf J}+ {1\over c^2}
{\partial {\bf E} \over \partial t}, 
\end{align}
where in these equations $c$ is actually $c_{a.u.}={1\over \alpha}$ and all 
other quantities are measured in a.u..
}
\\

{\color{orange} The form of the microscopic Maxwell's equations
in the c.g.s-Gaussian system has been discussed in the previous text:
\begin{align}
\nabla \cdot {\bf E} &= 4 \pi \rho, \\
\nabla \times {\bf E} &= -{1\over c} {\partial {\bf B} \over \partial t}, \\
\nabla \cdot {\bf B} &= 0, \\
\nabla \times {\bf B} &= {4 \pi \over c} {\bf J}+ {1\over c}
{\partial {\bf E} \over \partial t}, 
\end{align}
where in these equations $c$ is actually $c_{cgs}$ and all 
other quantities are measured in c.g.s.-Gaussian units.
}

\newpage
{\color{coral}\section{Macroscopic Maxwell's equations}}
\color{black}

In the SI the macroscopic Maxwell's equations can be formally obtained 
first by averaging the microscopic equations and then dividing the 
macroscopic charge density into a free part and a
bound part $\rho=\rho_f+\rho_b$ with $\rho_b=-\nabla \cdot {\bf P}$
where ${\bf P}$ is the polarization.
Similarly the macroscopic current density is divided into a free part, a 
magnetization part, 
and a displacement part with ${\bf J}={\bf J}_f + {\bf J}_m + {\bf J}_d$ where
${\bf J}_m=\nabla \times {\bf M}$ and ${\bf J}_d={\partial {\bf P}\over 
\partial t}$.
Inserting these expressions we have:

\begin{tcolorbox}
\begin{align}
\nabla \cdot {\bf D} &= \rho_f, \\
\nabla \times {\bf E} &= -{\partial {\bf B} \over \partial t}, \\
\nabla \cdot {\bf B} &= 0, \\
\nabla \times {\bf H} &= {\bf J}_f+ {\partial {\bf D} \over \partial t}. 
\end{align}
\end{tcolorbox}

where ${\bf D}=\epsilon_0 {\bf E}+{\bf P}$ and ${\bf H}={{\bf B}\over \mu_0}-
{\bf M}$.
\\

{\color{web-blue} The form of the macroscopic Maxwell's equations
in a.u. has been discussed in the previous text.
We can write $\rho=\rho_f+\rho_b$ with $\rho_b=-\nabla \cdot {\bf P}$
where ${\bf P}$ is the polarization.
Similarly the current density can be divided into a free part, a magnetization
part,
and a displacement part with ${\bf J}={\bf J}_f + {\bf J}_m + {\bf J}_d$ where
${\bf J}_m=\nabla \times {\bf M}$ and ${\bf J}_d={\partial {\bf P}\over 
\partial t}$
We have:
\begin{align}
\nabla \cdot {\bf D} &= 4 \pi \rho_f, \\
\nabla \times {\bf E} &= -{\partial {\bf B} \over \partial t}, \\
\nabla \cdot {\bf B} &= 0, \\
\nabla \times {\bf H} &= 4 \pi {\bf J}_f+ {\partial {\bf D} \over 
\partial t}. 
\end{align}
where ${\bf D}={\bf E} + 4 \pi {\bf P}$ and ${\bf H}={c^2
{\bf B}}- 4 \pi {\bf M}$.
Here $c$ is $c_{a.u.}={1\over \alpha}$ and all other quantities are 
measured in a.u..
}
\\

{\color{orange} The form of the macroscopic Maxwell's equations
in the c.g.s-Gaussian system has been discussed in the previous text.
In these units $\rho=\rho_f+\rho_b$ with $\rho_b=-\nabla \cdot {\bf P}$
where ${\bf P}$ is the polarization.
Similarly the current density can be divided into a free part, a 
magnetization part,
and a displacement part with ${\bf J}={\bf J}_f + {\bf J}_m + {\bf J}_d$ where
${\bf J}_m=c \nabla \times {\bf M}$ and ${\bf J}_d={\partial {\bf P}\over 
\partial t}$
\begin{align}
\nabla \cdot {\bf D} &= 4 \pi \rho_f, \\
\nabla \times {\bf E} &= -{1\over c} {\partial {\bf B} \over \partial t}, \\
\nabla \cdot {\bf B} &= 0, \\
\nabla \times {\bf H} &= {4 \pi \over c} {\bf J}_f+ {1\over c}
{\partial {\bf D} \over \partial t}, 
\end{align}
where ${\bf D}={\bf E} + 4 \pi {\bf P}$ and ${\bf H}={\bf B}- 4 \pi {\bf M}$.
In these equations $c$ is actually $c_{cgs}$ and all 
other quantities are measured in c.g.s.-Gaussian units.
}

\newpage

{\color{dark-blue}\chapter{Quantum Mechanics}}

{\color{coral}\section{The Schr\"odinger equation}}
\color{black}

In the SI the time independent Schr\"odinger equation for a particle 
with charge $q$, mass $m$ and spin $1/2$ in an electromagnetic field 
described by the scalar potential $\phi({\bf r})$ and the vector
potential ${\bf A}({\bf r})$ is:

\begin{tcolorbox}
\begin{equation}
\left[{1\over 2m} ({\bf p}-q {\bf A})^2 + q \phi - 
{q \hbar \over 2 m} {\boldsymbol \sigma} \cdot {\bf B} \right]
\Psi({\bf r})=E \Psi({\bf r})
\end{equation}
\end{tcolorbox}

where ${\bf B}=\nabla\times {\bf A}$, ${\boldsymbol \sigma}$ are the Pauli
matrices and $\Psi({\bf r})$ is a two component spinor.
\\

{\color{web-blue} In a.u. the time independent Schr\"odinger equation
becomes:
\begin{equation}
\left[{1\over 2m {\rm \bar m}} ({\rm \bar p} {\bf p}-{\rm \bar C}\cdot {\rm \bar A} q {\bf A})^2 + 
{\rm \bar C} \cdot {\rm \bar V} q \phi - 
{{\rm \bar C} \cdot {\rm \bar B} \over {\rm \bar m}} {\hbar q \over 2 m} 
{\boldsymbol \sigma} \cdot {\bf B} \right]
\Psi({\bf r})={\rm \bar E} E \Psi({\bf r})
\label{segen}
\end{equation}
Using ${\rm \bar p}={\hbar \over a_B}$, ${\rm \bar A}={\hbar \over e a_B}$, 
${\rm \bar C}=e$, ${\rm \bar V}= {E_h \over e}$, ${\rm \bar B}={\hbar \over e a_B^2}$ and
${\rm \bar E}=E_h$
we get
\begin{equation}
\left[{\hbar^2 \over m_e a_B^2} {1\over 2m} ({\bf p}-q {\bf A})^2 + 
E_h q \phi - {\hbar^2 \over m_e a_B^2} {q \over 2 m} 
{\boldsymbol \sigma} \cdot {\bf B} \right]
\Psi({\bf r})=E_h E \Psi({\bf r})
\end{equation}
but since $E_h= {\hbar^2 \over m_e a_B^2}$ the equation can be simplified into
\begin{equation}
\left[{1\over 2m} ({\bf p}-q {\bf A})^2 + 
q \phi - {q \over 2 m} 
{\boldsymbol \sigma} \cdot {\bf B} \right]
\Psi({\bf r})= E \Psi({\bf r})
\end{equation}
where all quantities are in a.u..\\

For an electron 
$m=1$ and $q=-1$ and one should write:
\begin{equation}
\left[{1\over 2} ({\bf p}+ {\bf A})^2  
- \phi + {1 \over 2} 
{\boldsymbol \sigma} \cdot {\bf B} \right]
\Psi({\bf r})= E \Psi({\bf r}).
\end{equation}
However this form is rarely used. One prefers to introduce the potential
energy of the electron $V=q\phi$ and keep the symbol $\mu_B=-{1\over 2}$ 
so that the equation is written:
\begin{equation}
\left[{1\over 2} ({\bf p}+ {\bf A})^2  
+ V - \mu_B {\boldsymbol \sigma} \cdot {\bf B} \right]
\Psi({\bf r})= E \Psi({\bf r}).
\end{equation}
When ${\bf A}=0$ we can write explicitly ${\bf p}=-i \nabla$, and
the Schr\"odinger equation becomes:
\begin{equation}
\left[-{1\over 2} \nabla^2  
+ V - \mu_B {\boldsymbol \sigma} \cdot {\bf B} \right]
\Psi({\bf r})= E \Psi({\bf r}).
\end{equation}
}
\\
{\color{orange} The form of the Schr\"odinger equation in the c.g.s.-Gaussian
system can be found starting from Eq.~\ref{segen} and using:
${\rm \bar p_{cgs}}=10^{-5} {{\rm kg}\cdot {\rm m} \over {\rm s}}$, ${\rm \bar C_{cgs}}\cdot {\rm \bar A_{cgs}}=
{1\over \mathcal{K}} {10^{-2}\over \mathcal{K}_T}\ {\rm C}\cdot {\rm T}\cdot {\rm m} = {10^{-5} \over \tilde c_{cgs}}
{{\rm kg}\cdot {\rm m} \over {\rm s}}$, ${\rm \bar C_{cgs}}\cdot {\rm \bar V_{cgs}}= {1\over \mathcal{K}} 10^{-7} \mathcal{K}\  
{\rm C}\cdot {\rm V}= 10^{-7} {\rm J}$, ${{\rm \bar C_{cgs}} \cdot {\rm \bar B_{cgs}} \hbar \over {\rm \bar m_{cgs}}}
= \tilde \hbar {1\over \mathcal{K}} {1\over \mathcal{K}_T 10^{-3}} {\rm J}={10^{-7} \tilde \hbar_{cgs}
\over \tilde c_{cgs}} {\rm J}$, and ${\rm \bar E_{cgs}}=10^{-7} {\rm J}$ leading to:
\begin{equation}
\left[{1\over 2m} ({\bf p}- {q \over c} {\bf A})^2 + 
q \phi - 
{\hbar q \over 2 m c} 
{\boldsymbol \sigma} \cdot {\bf B} \right]
\Psi({\bf r})= E \Psi({\bf r})
\end{equation}
where all quantities are expressed in c.g.s.-Gaussian units.
}

\newpage
\appendix

{\color{dark-blue}\chapter{Rydberg atomic units}}
{\color{violet}
In this Appendix we discuss the Rydberg a.u. using the same 
symbols used for Hartree a.u. with the subscript $R$.
Rydberg a.u. are defined requiring that
$a_B={\rm \bar l_R}$, $m_e={1\over 2}\ {\rm \bar m_R}$, $e=\sqrt{2}\ {\rm \bar C_R}$ and 
$\hbar= {{\rm \bar m_R} \cdot {\rm \bar l_R}^2 \over {\rm \bar t_R}}$.
We have therefore 
\begin{equation}
{\rm \bar l_R}=a_B={\rm \bar l},
\end{equation}
\begin{equation}
{\rm \bar m_R}=2 m_e=2 {\rm \bar m}=\barmry\ {\rm kg},
\end{equation}
\begin{equation}
{\rm \bar C_R}={e \over \sqrt{2}}= {{\rm \bar C} \over \sqrt{2}}=\barcry\ {\rm C},
\end{equation}
\begin{equation}
{\rm \bar t_R}={{\rm \bar m_R}\cdot {\rm \bar l_R}^2 \over \hbar}= {2 m_e a_B^2 \over \hbar}=
2 {\rm \bar t}=\bartry\ {\rm s}.
\end{equation}
Using these relationships we can derive the conversion factors for
the other units discussed in the text. \\
{\bf Frequency:}
\begin{equation}
\bar \nu_R={1\over {\rm \bar t_R}}= {1\over 2 {\rm \bar t}}= {1\over 2}\bar \nu=
\barnury\ {\rm Hz}.
\end{equation}
{\bf Speed:}
\begin{equation}
{\rm \bar v_R}={{\rm \bar l_R}\over {\rm \bar t_R}}= {{\rm \bar l}\over 2 {\rm \bar t}}= {1\over 2}{\rm \bar v}=
\barvry\ {\rm m}/{\rm s}.
\end{equation}
{\bf Acceleration:}
\begin{equation}
{\rm \bar a_R}={{\rm \bar l_R}\over {\rm \bar t_R}^2}= {{\rm \bar l}\over 4 {\rm \bar t}^2}=
{1\over 4}{\rm \bar a}=\barary\ {\rm m}/{\rm s}^2.
\end{equation}
{\bf Momentum:}
\begin{equation}
{\rm \bar p_R}={{\rm \bar m_R}\cdot {\rm \bar l_R}\over {\rm \bar t_R}}= {2 {\rm \bar m}\cdot {\rm \bar l}\over 2 {\rm \bar t}}= 
{\rm \bar p}.
\end{equation}
{\bf Angular momentum:}
\begin{equation}
{\rm \bar L_R}={{\rm \bar m_R}\cdot {\rm \bar l_R}^2\over {\rm \bar t_R}}= {2 {\rm \bar m}\cdot {\rm \bar l}^2\over 2 
{\rm \bar t}}= {\rm \bar L}.
\end{equation}
{\bf Force:}
\begin{equation}
{\rm \bar f_R}={{\rm \bar m_R}\cdot {\rm \bar l_R}\over {\rm \bar t_R}^2}= {2 {\rm \bar m}\cdot {\rm \bar l}\over 4 
{\rm \bar t}^2}= {1\over 2} {\rm \bar f}=\barfry\ {\rm N}.
\end{equation}
{\bf Energy:}
\begin{equation}
{\rm \bar U_R}={{\rm \bar f_R}\cdot {\rm \bar l_R}}= {1\over 2} {\rm \bar f}\cdot {\rm \bar l}
= {1\over 2} {\rm \bar U}=\barury\ {\rm J}.
\end{equation}
{\bf Power:}
\begin{equation}
{\rm \bar W_R}={{\rm \bar U_R} \over {\rm \bar t_R}}= {{\rm \bar U} \over 4 {\rm \bar t}}
= {1\over 4} {\rm \bar W}=\barwry\ {\rm W}.
\end{equation}
{\bf Pressure:}
\begin{equation}
{\rm \bar \sigma_R}={{\rm \bar f_R} \over {\rm \bar l_R}^2}= {{\rm \bar U} \over 2 {\rm \bar l}^2}
= {1\over 2} {\rm \bar \sigma}=\barprry\ {\rm Pa}.
\end{equation}
{\bf Current:}
\begin{equation}
{\rm \bar I_R}={{\rm \bar C_R} \over {\rm \bar t_R}}= {{\rm \bar C} \over 2 \sqrt{2}\ {\rm \bar t}}
= {1\over 2 \sqrt{2}} {\rm \bar I}=\bariry\ {\rm A}.
\end{equation}
{\bf Charge density:}
\begin{equation}
{\rm \bar \rho_R}={{\rm \bar C_R} \over {\rm \bar l_R}^3}= {{\rm \bar C} \over \sqrt{2}\ {\rm \bar l}^3}
= {1\over \sqrt{2}} {\rm \bar \rho}=\barrhory\ {\rm C}/{\rm m}^3.
\end{equation}
{\bf Current density:}
\begin{equation}
{\rm \bar J_R}={{\rm \bar I_R} \over {\rm \bar l_R}^2}= {{\rm \bar I} \over 2 \sqrt{2}\ {\rm \bar l}^2}
= {1\over 2 \sqrt{2}} {\rm \bar J}=\barcurry\ {\rm A}/{\rm m}^2.
\end{equation}
{\bf Electric field:}
\begin{equation}
{\rm \bar E_R}={{\rm \bar f_R} \over {\rm \bar C_R}}= {\sqrt{2}\ {\rm \bar f} \over 2 {\rm \bar C}}
= {1\over \sqrt{2}} {\rm \bar E}=\barery\ {\rm N}/{\rm C}.
\end{equation}
{\bf Electric potential:}
\begin{equation}
{\rm \bar V_R}={{\rm \bar E_R} \cdot {\rm \bar l_R}}= {{\rm \bar E} \over \sqrt{2}\ {\rm \bar l}}
= {1\over \sqrt{2}} {\rm \bar V}=\barphiry\ {\rm V}.
\end{equation}
{\bf Capacitance:}
\begin{equation}
{\rm \bar F_R}={{\rm \bar C_R} \over {\rm \bar V_R}}= {\sqrt{2}\ {\rm \bar C} \over \sqrt{2}\ {\rm \bar V}}
= {\rm \bar F}.
\end{equation}
{\bf Electric dipole moment:}
\begin{equation}
{\rm \bar \wp_R}={{\rm \bar C_R} \cdot {\rm \bar l_R}}= {1\over \sqrt{2}} {\rm \bar C} \cdot {\rm \bar l} 
= {1\over \sqrt{2}} {\rm \bar \wp}=\bardipry\ {\rm C}\cdot {\rm m}.
\end{equation}
{\bf Polarization:}
\begin{equation}
{\rm \bar P_R}={{\rm \bar C_R} \over {\rm \bar l_R}^2}= {{\rm \bar C} \over \sqrt{2}\ {\rm \bar l}^2}
= {1\over \sqrt{2}} {\rm \bar P}=\barpolarry\ {\rm C}/{\rm m}^2.
\end{equation}
{\bf Resistence:}
\begin{equation}
\bar \Omega_R={{\rm \bar V_R} \over {\rm \bar I_R}}= {2 \sqrt{2}\ {\rm \bar V} \over \sqrt{2}\  
{\rm \bar I}}
= 2 \bar \Omega=\barohmry\ \Omega.
\end{equation}
{\bf Magnetic flux density:}
\begin{equation}
{\rm \bar B_R}={{\rm \bar f_R} \over {\rm \bar C_R} \cdot {\rm \bar v_R}}= {2 \sqrt{2}\ {\rm \bar f} \over 2 
{\rm \bar C}\cdot {\rm \bar v}}
= \sqrt{2} {\rm \bar B}=\barbry\ {\rm T}.
\end{equation}
{\bf Vector potential:}
\begin{equation}
{\rm \bar A_R}={{\rm \bar B_R}\cdot {\rm \bar l_R} }= \sqrt{2}\ {\rm \bar B}
\cdot {\rm \bar l} 
= \sqrt{2}\ {\rm \bar A}=\baravry\  {\rm T}\cdot {\rm m}.
\end{equation}
{\bf Magnetic field flux:}
\begin{equation}
{\rm \bar Wb_R}={{\rm \bar B_R} \cdot {\rm \bar l_R}^2 }= \sqrt{2}\ {\rm \bar B} \cdot {\rm \bar l}^2
= \sqrt{2}\ {\rm \bar Wb}=\barwbry\ {\rm Wb}.
\end{equation}
With this definition the Lorentz force remains:
\begin{equation}
{\bf F}=q {\bf E} + q \left({\bf v}\times {\bf B}\right),
\end{equation}
and since ${{\rm \bar l_R}\cdot {\rm \bar B_R} \over {\rm \bar E_R}\cdot {\rm \bar t_R}}={{\rm \bar l_R}\cdot {\rm \bar B_R} \over {\rm \bar E_R}\cdot {\rm \bar t_R}}=1$ the second Maxwell's equation remains:
\begin{equation}
\nabla\times {\bf E}=-{\partial {\bf B}\over \partial t},
\end{equation}
{\bf Inductance:}
\begin{equation}
{\rm \bar Y_R}={{\rm \bar Wb_R} \over {\rm \bar I_R} }= 2 (\sqrt{2})^2\ {{\rm \bar Wb} \over {\rm \bar I}}
= 4 {\rm \bar Y}=\baryry\ {\rm H}.
\end{equation}
{\bf Magnetic dipole moment:}
\begin{equation}
{\rm \bar \mu_R}={{\rm \bar I_R}\cdot {\rm \bar l_R}^2}= {1\over 2\sqrt{2}}\ {{\rm \bar I}\cdot {\rm \bar l}^2}
= {1\over 2\sqrt{2}} {\rm \bar \mu}=\barmury\ {\rm A}\cdot {\rm m}^2.
\end{equation}
{\bf Magnetization:}
\begin{equation}
{\rm \bar M_R}={{\rm \bar I_R} \over {\rm \bar l_R}}= {1\over 2\sqrt{2}} {{\rm \bar I}\cdot {\rm \bar l}}
= {1\over 2\sqrt{2}} {\rm \bar M}=\barmagry\ {\rm A}/{\rm m}.
\end{equation}

In order to discuss the units of the electric displacement and of
the magnetic field strength, we need to discuss the Maxwell's equations.
The first Maxwell's equation is
\begin{equation}
\nabla\cdot {\bf E} = 4 \pi \rho,
\end{equation}
since ${{\rm \bar \rho_R}\cdot {\rm \bar l_R} \over \epsilon_0 {\rm \bar E_R}}=
{{\rm \bar \rho}\cdot {\rm \bar l} \over \epsilon_0 {\rm \bar E}}=4\pi$. Moreover since
${{\rm \bar P_R} \over {\rm \bar l_R}\cdot {\rm \bar \rho_R}}={{\rm \bar P} \over {\rm \bar l}\cdot {\rm \bar \rho}}=1$
in these units 
\begin{equation}
\rho_b=-\nabla \cdot {\bf P},
\end{equation} 
and the macroscopic Maxwell's equation becomes:
\begin{equation}
\nabla\cdot ({\bf E}+4\pi {\bf P}) = 4 \pi \rho_f,
\end{equation}
suggesting the definition:
\begin{equation}
{\bf D}={\bf E}+4\pi {\bf P}.
\end{equation}
Therefore:
\begin{equation}
{\rm \bar D_R}={{\rm \bar P_R} \over 4 \pi} = {1\over \sqrt{2}} {\rm \bar D}=\bardry\ {\rm C}/{\rm m}^2.
\end{equation}
The fourth Maxwell's equation can be written using
${\mu_0 {\rm \bar J_R}\cdot {\rm \bar l_R} \over {\rm \bar B_R}}={1\over 4} 
{\mu_0 {\rm \bar J}\cdot {\rm \bar l} \over {\rm \bar B}}= \pi \alpha^2$ and
${{\rm \bar E_R}\cdot {\rm \bar l_R} \over c^2 {\rm \bar t_R}\cdot {\rm \bar B_R}}={1\over 4}
{{\rm \bar E}\cdot {\rm \bar l} \over c^2 {\rm \bar t}\cdot {\rm \bar B}}={\alpha^2 \over 4}$ so we have:
\begin{equation}
\nabla \times {\bf B}= \pi \alpha^2 {\bf J} + {\alpha^2 \over 4}
{\partial {\bf E} \over \partial t}.
\end{equation}
Since ${{\rm \bar M_R} \over {\rm \bar l_R}\cdot {\rm \bar J_R}}={{\rm \bar M} \over {\rm \bar l}\cdot {\rm \bar J}}=1$
and ${{\rm \bar P_R} \over {\rm \bar t_R}\cdot {\rm \bar J_R}}={{\rm \bar P} \over {\rm \bar t}\cdot {\rm \bar J}}=1$
we can write
\begin{equation}
{\bf J}_m=\nabla \times {\bf M}
\end{equation}
and
\begin{equation}
{\bf J}_d={\partial {\bf P} \over \partial t},
\end{equation}
obtaining the equation
\begin{equation}
\nabla \times ({4{\bf B} \over \alpha^2}-4 \pi {\bf M})= 4 \pi 
{\bf J}_f + {\partial ({\bf E}+4\pi {\bf P}) \over \partial t}.
\end{equation}
Defining the magnetic field strengh as:
\begin{equation}
{\bf H}={4{\bf B} \over \alpha^2}-4 \pi {\bf M},
\end{equation}
we can write the fourth macroscopic Maxwell's equation as:
\begin{equation}
\nabla \times {\bf H}= 4 \pi 
{\bf J}_f + {\partial {\bf D} \over \partial t}.
\end{equation}
Therefore we have:
\begin{equation}
{\rm \bar H_R} ={{\rm \bar M_R} \over 4 \pi}={1\over \sqrt{2}} {{\rm \bar M} \over 4 \pi}=
{1\over \sqrt{2}} {\rm \bar H}=\barhry\ {\rm A}/{\rm m}.
\end{equation}
We note also that ${{\rm \bar l_R}\cdot {\rm \bar B_R} \over {\rm \bar E_R}\cdot {\rm \bar t_R}}=
{{\rm \bar l}\cdot {\rm \bar B} \over {\rm \bar E}\cdot {\rm \bar t}}=1$ so the second Maxwell's equation
remains:
\begin{equation}
\nabla \times {\bf E}=-{\partial {\bf B} \over \partial t}.
\end{equation}
Finally we consider the continuity equation. Since
${{\rm \bar t_R}\cdot {\rm \bar J_R} \over {\rm \bar \rho_R}\cdot {\rm \bar l_R}}={{\rm \bar t}\cdot {\rm \bar J} 
\over {\rm \bar \rho}\cdot {\rm \bar l}}=1$ this equation remains:
\begin{equation}
{\partial \rho \over \partial t}=-\nabla \cdot {\bf J}.
\end{equation}

The form of the Schr\"odinger equation in Rydberg a.u.
can be found starting from Eq.~\ref{segen} and using:
${\rm \bar p_R}={\rm \bar p}$, ${\rm \bar C_R}\cdot {\rm \bar A_R}= 
{\rm \bar C}\cdot {\rm \bar A}$,
${\rm \bar C_R}\cdot {\rm \bar V_R}= {1\over 2} {\rm \bar C}\cdot
{\rm \bar V}$,  
${{\rm \bar C_R} \hbar {\rm \bar B_R} \over {\rm \bar m_R}}
= {1\over 2} {{\rm \bar C} \hbar {\rm \bar B} \over {\rm \bar m}}$,
and ${\rm \bar E_R}={1\over 2} {\rm \bar E}$ leading to:
\begin{equation}
\left[{1\over 2m} ({\bf p}- q {\bf A})^2 + 
q \phi - 
{q \over 2 m} 
{\boldsymbol \sigma} \cdot {\bf B} \right]
\Psi({\bf r})= E \Psi({\bf r})
\end{equation}
where all quantities are expressed in Rydberg a.u.. \\

For an electron 
$m={1\over 2}$ and $q=-\sqrt{2}$ and one should write:
\begin{equation}
\left[({\bf p}+ \sqrt{2} {\bf A})^2  
- \sqrt{2} \phi + \sqrt{2}
{\boldsymbol \sigma} \cdot {\bf B} \right]
\Psi({\bf r})= E \Psi({\bf r}).
\end{equation}
However this form is rarely used. One prefers to introduce the potential
energy of the electron $V=q\phi$ and keep the symbol $\mu_B={q \over 2 m}
=-\sqrt{2}$ 
so that the equation is written:
\begin{equation}
\left[({\bf p}+ \sqrt{2} {\bf A})^2  
+ V - \mu_B {\boldsymbol \sigma} \cdot {\bf B} \right]
\Psi({\bf r})= E \Psi({\bf r}).
\end{equation}
When ${\bf A}=0$ we can write explicitly ${\bf p}=-i \nabla$, and
the Schr\"odinger equation becomes:
\begin{equation}
\left[-\nabla^2  
+ V - \mu_B {\boldsymbol \sigma} \cdot {\bf B} \right]
\Psi({\bf r})= E \Psi({\bf r}).
\end{equation}
}
\newpage

{\color{dark-blue}\chapter{Gaussian atomic units}}

{\color{steelblue}
It is possible to make the electromagnetic equations in a.u. look like 
those of the c.g.s.-Gaussian system. This requires the modification of 
the units of the magnetic flux density, vector potential, magnetic field 
flux, magnetic dipole moment, magnetization, and magnetic field strength, 
while the other units remain unchanged. 
We discuss these quantities here using a subscript $G$ to identify the
modified units. Although the unit of inductance is not modified 
we discuss it since it depends on the definition of the magnetic field flux.
In order to avoid confusion we use $\alpha$ which is the same in all
systems, but these equations are sometimes written using the speed of light
$c={1\over \alpha}$ instead of $\alpha$.
\\

\noindent {\bf Magnetic flux density:\\}
The definition of the Lorentz force in these units should be:
\begin{equation}
{\bf F}=\alpha q ({\bf v}\times {\bf B}).
\end{equation}
Therefore we must have
${{\rm \bar C_G}\cdot {\rm \bar v_G}\cdot {\rm \bar B_G} \over {\rm \bar f_G}} = \alpha$. Since one
takes ${\rm \bar C_G}={\rm \bar C}$, ${\rm \bar v_G}={\rm \bar v}$ and ${\rm \bar f_G}={\rm \bar f}$, we have 
\begin{equation}
{\rm \bar B_G}=\alpha {\hbar \over e a_B^2}=\alpha {\rm \bar B}=\barbg\ {\rm T}.
\end{equation}
With this conversion factor we have ${{\rm \bar l_G}\cdot {\rm \bar B_G} \over {\rm \bar E_G}\cdot
{\rm \bar t_G}}=\alpha {{\rm \bar l}\cdot {\rm \bar B} \over {\rm \bar E}
\cdot {\rm \bar t}}=\alpha$ and the second Maxwell's equation becomes:
\begin{equation}
\nabla \times {\bf E}=-\alpha {\partial {\bf B} \over \partial t}.
\end{equation}\\
{\bf Vector potential:\\}
We still have ${\bf B}=\nabla \times {\bf A}$, therefore:
\begin{equation}
{\rm \bar A_G}={\rm \bar B_G}\cdot {\rm \bar l_G}=\alpha {\rm \bar A}= \baravg\ {\rm T} \cdot {\rm m}
\end{equation}
When the vector potential is time dependent it gives rise to an electric field
given by:
\begin{equation}
{\bf E}=-\nabla \Phi - \alpha {\partial {\bf A} \over \partial t}
\end{equation}
since ${{\rm \bar A_G} \over {\rm \bar t_G}\cdot {\rm \bar E_G}}=\alpha
{{\rm \bar A} \over {\rm \bar t}\cdot {\rm \bar E}}=\alpha$.\\

\noindent {\bf Magnetic field flux:\\}
We still have:
\begin{equation}
{\rm \bar \Phi_G}={\rm \bar B_G}\cdot {\rm \bar l_G}^2=\alpha \bar \Phi=\barwbg\ {\rm Wb}.
\end{equation}\\
{\bf Inductance:\\}
To have the same equation as in the c.g.s.-Gaussian system:
\begin{equation}
L=\alpha {\Phi \over I},
\end{equation} 
we must have ${{\rm \bar \Phi_G} \over {\rm \bar I_G}\cdot {\rm \bar Y_G}} = \alpha$ or: 
\begin{equation}
{\rm \bar Y_G}={{\rm \bar \Phi_G} \over \alpha {\rm \bar I_G}}={\rm \bar Y}.
\end{equation}\\
{\bf Magnetic dipole moment:\\}
To have the same equation as in the c.g.s.-Gaussian system:
\begin{equation}
{\boldsymbol \mu}=\alpha I A \hat {\bf n},
\label{magdipaucgs}
\end{equation}
we must have ${{\rm \bar I_G}\cdot {\rm \bar l_G}^2 \over {\rm \bar \mu_G}}=\alpha$ or
\begin{equation}
{\rm \bar \mu_G}={{\rm \bar I_G}\cdot {\rm \bar l_G}^2 \over \alpha} ={1 \over \alpha} \bar \mu= 
\barmug\ {\rm A}\cdot {\rm m}^2.
\end{equation}\\
{\bf Magnetization:\\}
We still have:
\begin{equation}
{\rm \bar M_G}={{\rm \bar \mu_G} \over {\rm \bar l_G}^3}={1\over \alpha} {\rm \bar M}=
\barmagg\ {\rm A}/{\rm m}.
\end{equation}\\
{\bf Magnetic field strength:\\}
The unit of the magnetic field strength is derived from the relationship
between magnetic field flux, magnetization, and magnetic field strength.
Since ${\mu_0 {\rm \bar J_G}\cdot {\rm \bar l_G} \over {\rm \bar B_G}}=
{\mu_0 {\rm \bar J}\cdot {\rm \bar l} \over \alpha {\rm \bar B}}=4\pi \alpha$ and
${{\rm \bar E_G}\cdot {\rm \bar l_G} \over c^2 {\rm \bar t_G}\cdot {\rm \bar B_G}}=
{{\rm \bar E}\cdot {\rm \bar l} \over c^2 {\rm \bar t} \alpha {\rm \bar B}}=\alpha$, 
the fourth Maxwell's equation becomes
\begin{equation}
\nabla \times {\bf B}= 4 \pi \alpha {\bf J} + \alpha
{\partial {\bf E} \over \partial t}.
\end{equation}
Since ${{\rm \bar M_G} \over {\rm \bar l_G}\cdot {\rm \bar J_G}}=
{{\rm \bar M} \over \alpha {\rm \bar l}\cdot {\rm \bar J}}=
{1\over \alpha}$ and 
${{\rm \bar P_G} \over {\rm \bar t_G}\cdot {\rm \bar J_G}}={{\rm \bar P} \over {\rm \bar t}\cdot {\rm \bar J}}=1$ we
can write:
\begin{equation}
{\bf J}_m={1\over \alpha} \nabla \times {\bf M}
\end{equation}
and 
\begin{equation}
{\bf J}_d={\partial {\bf P} \over \partial t}.
\end{equation}
So the fourth macroscopic Maxwell's equation becomes
\begin{equation}
\nabla \times ({\bf B}-4\pi {\bf M})= 4 \pi \alpha {\bf J}_f + \alpha
{\partial ({\bf E}+4\pi {\bf P}) \over \partial t}.
\end{equation}
This suggests the definition of the magnetic field strength as
${\bf H}={\bf B}-4 \pi {\bf M}$ in addition to the usual equation 
for the electric induction ${\bf D} = {\bf E}+4\pi {\bf P}$. 
This gives:
\begin{equation}
\nabla \times {\bf H}= 4 \pi \alpha {\bf J}_f + \alpha
{\partial {\bf D} \over \partial t}.
\end{equation}
The unit of magnetic field strength is therefore derived from
${\mu_0 {\rm \bar B_G} \over {\rm \bar H_G}}=1$ or ${{\rm \bar M_G} \over {\rm \bar H_G}}=4 \pi$ that
gives
\begin{equation}
{\rm \bar H_G} = \mu_0 {\rm \bar B_G}=\alpha \mu_0 {\rm \bar B}={{\rm \bar M_G} \over 4 \pi}= 
{1\over \alpha} {\rm \bar H}=\barhg\ {\rm A}/{\rm m}.
\end{equation}
}
\newpage

{\color{dark-blue}\chapter{Magnetization Intensity}}

Several authors use the magnetization intensity instead of the
magnetization. This quantity is defined as:
\begin{equation}
{\bf I}=\mu_0 {\bf M},
\end{equation}
and has the same unit of the magnetic field (Tesla $T$). 
The relationship between magnetic field strength, magnetic flux
density, and magnetization intensity is written as:
\begin{equation}
{\bf B}= \mu_0 {\bf H} + {\bf I}.
\end{equation} 

{\color{dark-blue}\chapter{Conversion factors tables}}

This Appendix collects the conversion factors discussed in the text
and gives their absolute and relative errors. 
The data reported here have been obtained with the tool 
\texttt{tools/units.f90} using as input the values of the first seven
quantities of this list taken from the NIST web-site 
(\texttt{https://www.nist.gov/}). 
Errors are calculated from the errors of the Rydberg constant and of
the fine structure constant available in the same site. \\
These values are updated to October 2019.

\begin{tcolorbox}
\begin{verbatim}
Experimental quantities exact in the SI:
Planck constant:           6.62607015E-34      J.s
Planck constant / 2 pi:    1.0545718176462E-34 J.s
Speed of light:            2.99792458E+08      m/s
Electron charge:           1.602176634E-19     C
Avogadro number:           6.02214076E+23
Boltzmann constant:        1.380649E-23        J/K

Approximate quantities determined by experiment:
Rydberg constant:          1.0973731568160E+07 1/m
Fine structure constant:   7.2973525693E-03
Atomic mass unit:          1.66053906660E-27   kg

Derived Physical quantities:
Electron mass:             9.1093837015E-31    kg
mu0:                       1.25663706212E-06   N/A^2
epsilon0:                  8.8541878128E-12    C^2/N m^2
E_hatree:                  4.3597447222072E-18 J
Bohr radius:               5.29177210903E-11   m
Bohr magneton:             9.2740100783E-24    J/T

Conversion factors (Atomic units - SI):
Length: \l=                5.29177210903E-11   m
Mass: \m=                  9.1093837015E-31    kg
Mass density: \rhom=       6.1473168257E+00    kg/m^3
Time: \t=                  2.4188843265857E-17 s
Frequency: \nu=            4.1341373335182E+16 Hz
Speed: \v=                 2.18769126364E+06   m/s
Acceleration: \a=          9.0442161272E+22    m/s^2
Momentum: \p=              1.99285191410E-24   kg m/s
Angular momentum: \L=      1.0545718176462E-34 kg m^2/s
Force: \f=                 8.2387234982E-08    N
Energy: \U=                4.3597447222072E-18 J
Power: \W=                 1.8023783420686E-01 W
Pressure: \pr=             2.9421015696E+13    Pa
Current: \I=               6.623618237510E-03  A
Charge: \C=                1.602176634E-19     C
Charge density: \rho=      1.08120238456E+12   C/m^3
Current density: \J=       2.36533701094E+18   A/m^2
Electric field: \E=        5.14220674763E+11   N/C
Electric potential: \V=    2.7211386245988E+01 V
Capacitance: \F=           5.887890530517E-21  F
Dipole moment: \dip=       8.4783536255E-30    C m
Polarization: \P=          5.7214766229E+01    C/m^2
Electric displacement: \D= 4.5530064316E+00    C/m^2
Resistance: \R=            4.1082359022277E+03 Ohm
Magnetic induction: \B=    2.35051756758E+05   T
Vector potential: \A=      1.24384033059E-05   T m
Magnetic field flux: \Phi= 6.5821195695091E-16 Wb
Inductance: \Y=            9.937347433815E-14  H
Magnetic dipole: \mu=      1.85480201567E-23   A m^2 (J/T)
Magnetization: \M=         1.25168244230E+08   A/m
Magnetic strength: \H=     9.9605723936E+06    A/m

Conversion factors (c.g.s.-Gaussian - SI):
Length: cm=                1.0E-02             m
Mass: g=                   1.0E-03             kg
Mass density: g/cm^3=      1.0E+03             kg/m^3
Time: s=                   1.0E+00             s
Frequency: Hz=             1.0E+00             Hz
Speed: cm/s=               1.0E-02             m/s
Acceleration: cm/s^2=      1.0E-02             m/s^2
Momentum: g cm/s=          1.0E-05             kg m/s
Angular momentum: g cm^2/s=1.0E-07             kg m^2/s
Force: dyne=               1.0E-05             N
Energy: erg=               1.0E-07             J
Power: erg/s=              1.0E-07             W
Pressure: Ba=              1.0E-01             Pa
Current: statA=            3.33564095107E-10   A
Charge: statC=             3.33564095107E-10   C
Charge density: statC/cm^3=3.33564095107E-04   C/m^3
Current density: statA/cm^2=3.33564095107E-06  A/m^2
Electric field: dyne/statC=2.99792458082E+04   N/C
Electric potential: statV= 2.99792458082E+02   V
Capacitance: cm=           1.11265005544E-12   F
Dipole moment: statC cm=   3.33564095107E-12   C m
Electric polarization: statC/cm^2= 3.33564095107E-06 C/m^2
Electric displ: statC/cm^2 4 pi = 2.65441872871E-07 C/m^2
Resistance: s/cm=          8.987551792288E+11  Ohm
Magnetic induction: G=     1.000000000274E-04  T
Vector potential: G cm=    1.000000000274E-06  T m
Magnetic field flux: Mx=   1.000000000274E-08  Wb
Inductance: statH=         8.98755179229E+11   H
Magnetic dipole: statC cm= 9.99999999726E-04   A m^2
Magnetization: statC/cm^2= 9.99999999726E+02   A/m
Magnetic strength: statC/cm^2 4 pi= 7.95774715242E+01 A/m

C/statC= 2.99792458082E+09 (C/statC)/10c= 1.00000000027E+00

mu_0/4 pi 10^-7=      1.00000000055E+00

Conversion factors (SI - c.g.s.-Gaussian):
Length: m=                 1.0E+02             cm
Mass: kg=                  1.0E+03             g
Time: s=                   1.0E+00             s
Frequency: Hz=             1.0E+00             Hz
Speed: m/s=                1.0E+02             cm/s
Acceleration: m/s^2=       1.0E+02             cm/s^2
Momentum: kg m/s=          1.0E+05             g cm/s
Angular momentum: kg m^2/s=1.0E+07             g cm^2/s
Force: N=                  1.0E+05             dyne
Energy: J=                 1.0E+07             erg
Power: W=                  1.0E+07             erg/s
Pressure: Pa=              1.0E+01             Ba
Current: A=                2.99792458082E+09   statA
Charge: C=                 2.99792458082E+09   statC
Charge density: C/m^3=     2.99792458082E+03   statC/cm^3
Current density: A/m^2=    2.99792458082E+05   statA/cm^2
Electric field: N/C=       3.33564095107E-05   dyne/statC
Electric potential: V=     3.33564095107E-03   statV
Capacitance: F=            8.98755179229E+11   cm
Dipole moment: C m=        2.99792458082E+11   statC cm
Electric polarization: C/m^2= 2.99792458082E+05 statC/cm^2
Electric displ.: C/m^2= 3.76730313565E+06 statC/cm^2 4pi
Resistance: Ohm=           1.112650055445E-12  s/cm
Magnetic induction: T=     9.999999997263E+03  G
Vector potential: T m=     9.999999997263E+05  G cm
Magnetic field flux: Wb=   9.999999997263E+07  Mx
Inductance: H=             1.11265005544E-12   statH
Magnetic dipole: A m^2=    1.00000000027E+03   statC cm
Magnetization: A/m=        1.00000000027E-03   statC/cm^2
Magnetic strength: A/m=    1.25663706178E-02   statC/cm^2 4pi

Conversion factors (Atomic units - c.g.s.-Gaussian):
Length:                    5.29177210903E-09   cm
Mass:                      9.1093837015E-28    g
Mass density:              6.1473168257E-03    g/cm^3
Time:                      2.4188843265857E-17 s
Frequency:                 4.1341373335182E+16 Hz
Speed:                     2.18769126364E+08   cm/s
Acceleration:              9.0442161272E+24    cm/s^2
Momentum:                  1.99285191410E-19   g cm/s
Angular momentum:          1.0545718176462E-27 g cm^2/s
Force:                     8.2387234982E-03    dyne
Energy:                    4.3597447222072E-11 erg
Power:                     1.8023783420686E+06 erg/s
Pressure:                  2.9421015696E+14    Ba
Current:                   1.98571079282E+07   statA
Charge:                    4.80320471388E-10   statC
Charge density:            3.2413632055E+15    statC/cm^3
Current density:           7.0911019670E+23    statA/cm^2
Electric field:            1.71525554062E+07   dyne/statC
Electric potential:        9.07674142975E-02   statV
Capacitance:               5.29177210903E-09   cm
Dipole moment:             2.54174647389E-18   statC cm
Polarization:              1.71525554062E+07   statC/cm^2
Electric displacement:     1.71525554062E+07   statC/cm^2 4pi
Resistance:                4.57102890439E-09   s/cm
Magnetic induction:        2.35051756693E+09   G
Vector potential:          1.24384033025E+01   G cm
Magnetic field flux:       6.58211956771E-08   Mx
Inductance:                1.10567901732E-25   statH
Magnetic dipole:           1.85480201618E-20   statC cm
Magnetization:             1.25168244264E+05   statC/cm^2
Magnetic strength:         1.25168244264E+05   statC/cm^2 4pi 

Conversion factors (Rydberg atomic units - SI):
Length: \l_R=              5.29177210903E-11   m
Mass: \m_R=                1.82187674030E-30   kg
Time: \t_R=                4.8377686531714E-17 s
Frequency: \nu_R=          2.0670686667591E+16 Hz
Speed: \v_R=               1.09384563182E+06   m/s
Acceleration: \a_R=        4.52210806362E+22   m/s^2
Momentum: \p_R=            1.99285191410E-24   kg m/s
Angular momentum: \L_R=    1.0545718176462E-34 kg m^2/s
Force: \f_R=               4.11936174912E-08   N
Energy: \U_R=              2.1798723611036E-18 J
Power: \W_R=               4.505945855171E-02  W
Pressure: \pr_R=           1.47105078482E+13   Pa
Current: \I_R=             2.3418026858671E-03 A
Charge: \C_R=              1.1329099625600E-19 C
Charge density: \rho_R=    7.6452553796E+11    C/m^3
Current density: \J_R=     8.36272920114E+17   A/m^2
Electric field: \E_R=      3.63608926151E+11   N/C
Electric potential: \V_R=  1.9241355740025E+01 V
Capacitance: \F_R=         5.887890530517E-21  F
Dipole moment: \dip_R=     5.99510134192E-30   C m
Polarization: \P_R=        4.0456949184E+01    C/m^2
Electric displacement: \D_R=3.21946172254E+00   C/m^2
Resistance: \R_R=          8.2164718044553E+03 Ohm
Magnetic induction: \B_R=  3.3241338227E+05    T
Vector potential: \A_R=    1.75905586495E-05   T m
Magnetic field flux: \Phi_R=9.3085227643611E-16 Wb
Inductance: \Y_R=          3.9749389735260E-13 H
Magnetic dipole: \mu_R=    6.5577154152E-24    A m^2 (J/T)
Magnetization: \M_R=       4.42536571420E+07   A/m
Magnetic field: \H_R=      3.52159414202E+06   A/m

Conversion factors (Gaussian atomic units - SI):
Magnetic induction: \B_G=  1.71525554109E+03   T
Vector potential: \A_G=    9.0767414322E-08    T m
Magnetic field flux: \Phi_G=4.80320471520E-18  Wb
Magnetic dipole: \mu_G=    2.5417464732E-21    A m^2 (J/T)
Magnetization: \M_G=       1.71525554016E+10   A/m
Magnetic field: \H_G=      1.36495698941E+09   A/m

Physical constants in Hartree atomic units:
Speed of light:            1.37035999084E+02
atomic mass unit:          1.8228884862E+03

Physical constants in eV:
Hartree in eV:             2.7211386246E+01
Rydberg in eV:             1.3605693123E+01

Frequency conversion:
Hz in cm^-1:               3.33564095198152E-11
cm^-1 in Hz:               2.99792458E+10

------------------------------------------------------------
Errors:                   Absolute                 Relative

rydberg =                  2.10E-05 1/m             1.91E-12
alpha =                    1.10E-12                 1.51E-10
amu =                      5.00E-37                 3.01E-10

me =                       2.76E-40 kg              3.03E-10
abohr =                    8.08E-21 m               1.53E-10
mu0 =                      1.89E-16 N/A^2           1.51E-10
epsilon0 =                 1.33E-21 C^2/Nm^2        1.51E-10
hartree =                  8.34E-30 J               1.91E-12
bohr mag =                 2.81E-33 J/T             3.03E-10

Errors of conversion factors (atomic units - SI):
\l =                       8.08E-21 m               1.53E-10
\m =                       2.76E-40 kg              3.03E-10
\rhom =                    4.68E-09 kg/m^3          7.61E-10
\t =                       4.63E-29 s               1.91E-12
\nu =                      7.91E+04 s               1.91E-12
\v =                       3.30E-04 m/s             1.51E-10
\a =                       1.38E+13 m/s^2           1.53E-10
\p =                       3.04E-34 kg m/s          1.53E-10
\L =                       0.00E+00 kg m^2/s        0.00E+00
\f =                       1.27E-17 N               1.55E-10
\U =                       8.34E-30 J               1.91E-12
\W =                       6.90E-13 W               3.83E-12
\pr =                      1.35E+04 Pa              4.60E-10
\I =                       1.27E-14 A               1.91E-12
\C =                       0.00E+00 C               0.00E+00
\rho =                     4.95E+02 C/m^3           4.58E-10
\J =                       7.27E+08 A/m^2           3.07E-10
\E =                       7.95E+01 V/m             1.55E-10
\V =                       5.21E-11 V               1.91E-12
\F =                       1.13E-32 F               1.91E-12
\dip =                     1.29E-39 C m             1.53E-10
\P =                       1.75E-08 C/m^2           3.05E-10
\D =                       1.39E-09 C/m^2           3.05E-10
\R =                       0.00E+00 Ohm             0.00E+00
\B =                       7.18E-05 T               3.05E-10
\A =                       1.90E-15 T m             1.53E-10
\phi =                     0.00E+00 Wb              0.00E+00
\Y =                       1.90E-25 H               1.91E-12
\mu =                      5.63E-33 J/T             3.03E-10
\M =                       1.93E-02 A/m             1.55E-10
\H =                       1.54E-03 A/m             1.55E-10

cspeed =                   2.07E-08 \v              1.51E-10
amu =                      1.10E-06 \m              6.04E-10


Errors of conversion factors (c.g.s.-Gaussian - SI):
Current =                  2.51E-20 A               7.54E-11
Charge =                   2.51E-20 C               7.54E-11
Charge density =           2.51E-14 C/m^3           7.54E-11
Current density =          2.51E-16 A/m^2           7.54E-11
Electric field =           2.26E-06 V/m             7.54E-11
Electric potential =       2.26E-08 V               7.54E-11
Capacitance =              1.68E-22 F               1.51E-10
Dipole moment =            2.51E-22 C m             7.54E-11
Electric Polarization =    2.51E-16 C/m^2           7.54E-11
Electric Displacement =    2.00E-17 C/m^2           7.54E-11
Resistance =               1.35E+02 Ohm             1.51E-10
Magnetic induction =       7.54E-15 T               7.54E-11
Vector potential =         7.54E-17 T m             7.54E-11
Magnetic field flux =      7.54E-19 Wb              7.54E-11
Inductance =               1.35E+02 H               1.51E-10
Magnetic dipole =          7.54E-14 J/T             7.54E-11
Magnetization =            7.54E-08 A/m             7.54E-11
Magnetic strength =        6.00E-09 A/m             7.54E-11


Errors of conversion factors (atomic units-c.g.s.-Gaussian):
Current =                  1.53E-03 statA           7.73E-11
Charge =                   3.62E-20 statC           7.54E-11
Charge density =           1.73E+06 statC/cm^3      5.33E-10
Current density =          2.71E+14 statA/cm^2      3.83E-10
Electric field =           3.94E-03 statV/cm        2.30E-10
Electric potential =       7.01E-12 statV           7.73E-11
Capacitance =              8.08E-19 cm              1.53E-10
Dipole moment =            5.80E-28 statC cm        2.28E-10
Electric Polarization =    6.53E-03 statC/cm^2      3.81E-10
Electric displacement =    6.53E-03 statC/cm^2 4pi  3.81E-10
Resistance =               6.89E-19 s/cm            1.51E-10
Magnetic induction =       8.95E-01 G               3.81E-10
Vector potential =         2.84E-09 G cm            2.28E-10
Magnetic field flux =      4.96E-18 Mx              7.54E-11
Inductance =               1.69E-35 statH           1.53E-10
Magnetic dipole =          7.03E-30 statC cm        3.79E-10
Magnetization =            2.88E-05 statC/cm^2      2.30E-10
Magnetic strength =        2.88E-05 statC/cm^2 4pi  2.30E-10

Errors of conversion factors (Rydberg atomic units - SI):
\l_R =                     8.08E-21 m               1.53E-10
\m_R =                     5.53E-40 kg              3.03E-10
\t_R =                     9.26E-29 s               3.83E-12
\nu_R =                    3.96E+04 s               1.91E-12
\v_R =                     1.65E-04 m/s             1.51E-10
\a_R =                     6.90E+12 m/s^2           1.53E-10
\p_R =                     3.04E-34 kg m/s          1.53E-10
\L_R =                     0.00E+00 kg m^2/s        0.00E+00
\f_R =                     6.37E-18 N               1.55E-10
\U_R =                     4.17E-30 J               1.91E-12
\W_R =                     1.72E-13 W               3.83E-12
\pr_R =                    6.76E+03 Pa              4.60E-10
\I_R =                     4.48E-15 A               1.91E-12
\C_R =                     0.00E+00 C               0.00E+00
\rho_R =                   3.50E+02 C/m^3           4.58E-10
\J_R =                     2.57E+08 A/m^2           3.07E-10
\E_R =                     5.62E+01 V/m             1.55E-10
\V_R =                     5.21E-11 V               2.71E-12
\F_R =                     1.13E-32 F               1.91E-12
\dip_R =                   9.15E-40 C m             1.53E-10
\P_R =                     1.24E-08 C/m^2           3.05E-10
\D_R =                     9.83E-10 C/m^2           3.05E-10
\R_R =                     0.00E+00 Ohm             0.00E+00
\B_R =                     1.01E-04 T               3.05E-10
\A_R =                     2.69E-15 T m             1.53E-10
\phi_R =                   0.00E+00 Wb              0.00E+00
\Y_R =                     7.61E-25 H               1.91E-12
\mu_R =                    1.99E-33 J/T             3.03E-10
\M_R =                     6.84E-03 A/m             1.55E-10
\H_R =                     5.44E-04 A/m             1.55E-10


Errors of conversion factors (Gaussian atomic units - SI):
\B_G =                     7.82E-07 T               4.56E-10
\A_G =                     2.75E-17 T m             3.03E-10
\phi_G =                   7.24E-28 Wb              1.51E-10
\mu_G =                    1.15E-30 J/T             4.54E-10
\M_G =                     5.24E+00 A/m             3.05E-10
\H_G =                     4.17E-01 A/m             3.05E-10
\end{verbatim}
\end{tcolorbox}
\newpage


{\color{dark-blue}\chapter{Bibliography}}
\begin{enumerate}

\item
[1.] J.D. Jackson, Classical electrodynamics, Third edition,
J. Wiley and Sons (1999).

\end{enumerate}

\end{document}

